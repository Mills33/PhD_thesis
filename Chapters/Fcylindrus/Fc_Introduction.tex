\section{Introduction}

\subsection{Sexual reproduction and recombination}

Sex as a mode of reproduction has a two-fold cost.
Firstly, most sexually reproducing species only have one gender capable of bearing offspring \parencite{DeVisser2007}.
Secondly, in sexually reproducing organisms, any individual will only contribute approximately half of its genetic information to each offspring; i.e. in diploid sexuals, gametes are haploid \parencite{Agrawal2001}.
In contrast, an asexually reproducing, clonal organism contributes all of its genetic information to each offspring, and every individual is typically capable of bearing young \parencite{Schlupp2010}.
This generalization applies to most sexual organisms however, there are exceptions. 
For example, not all sexually producing organisms have the two-fold cost problem. 
Yeasts are sexual organisms with two mating types and both types are capable of producing offspring.
In addition, a species of poecilids can reproduce through a process of gynogenesis; a process similar to asexual reproduction through parthenogenesis, but is distinct as the presence of sperm is required to stimulate egg development \parencite{Schlupp2010}.
Hybridisation has also given rise to a Hermaphroditic Cichlid individual which can self \parencite{Svensson2016}.
In addition, some species shuttle between asexual and sexual reproduction, and the frequency at which this happens directly affects the factors raised above.

All else being equal, an asexual species should outperform a sexual species over time because of its faster population growth rate.
However, sexual and asexual species do co-exist together, sometimes with similar fecundity \parencite{Schlupp2010}.
However, despite this, sexual reproduction is very widespread, especially among the eukaryotes.
These observations led researchers to think that the benefits of sexual reproductions must be evolutionary and lead to the production of offspring with benefits that outweigh to costs.
To summarize most of the commonly cited reasons sexual reproduction is maintained, it may be described as a mechanism, through which: 

\begin{enumerate}
\item Beneficial mutations can spread through a population more quickly.
\item Novel genetic combinations are generated.
\item Deleterious mutations can be purged or masked.
\end{enumerate}

These benefits are possible because sexual reproduction brings together into one individual, the chromatids (and alleles they contain) in the gametes of two parental individuals from separate genealogical lines (out-crossing).
In addition, when parental individuals generate gametes, meiotic recombination will result in new combinations of genes \parencite{Felsenstein1976}.
This in turn contributes to the generation of novel genetic (or rather, genotypic) variation.
As a result, two or more beneficial mutations from separate genealogical lines may occur together within the same individual, thus facilitating the spread of beneficial mutations through the population to fixation.

This is formalized by the Hill-Robertson effect \parencite{Hill1966}, and is demonstrated by considering two loci with the haplotype $A_2B_2$ with a fitness of $1$.
It is then assumed two mutants at both loci ($A_1B_2$, $A_2B_1$) can occur after a time period with fitnesses of $1+s$, and that fitnesses are multiplicative such that $A_1B_1$ has fitness $(1+s)^2$.
With no or low recombination, the ancestral haplotype is lost by selection and both advantageous mutants will exist in the population for some time until one is lost by drift \parencite{Coop2007}.
But with recombination, a haplotype $A_1B_1$ is possible, bringing both mutants together in one haplotype before one of the mutants is lost by drift, thus both mutants get fixed rather than one \cite{Coop2007}.
With low recombination rates selection increasing the frequency of the mutant alleles is less effective, this is the Hill-Robertson effect \parencite{Hill1966}.

The effect is more likely to occur when selection is not too strong, recombination rates are low, and when the favorable mutants have negative disequilibrium i.e. they initially occur on different haplotypes \parencite{Hedrick2010}.
An asexual lineage, in contrast would have to acquire one beneficial mutation, followed by another, a limitation called “clonal interference” \parencite{Gerrish1998}.

Similarly, deleterious mutations accumulating throughout the population in different genealogical lines may occur together within one individual, which suffers stronger negative selection pressure and is eliminated from the population \parencite{Crow1994}.
A third possibility is a deleterious allele is inherited from one parent, and the corresponding allele inherited from the second parent is not deleterious.
In that case, the affects of the deleterious allele may be alleviated or masked, as the offspring individual still possesses a non-deleterious copy.
Chromosomal crossover during meiosis may also result in the removal of deleterious mutations \parencite{Crow1994}.

The maintenance of sexual reproduction has also been attributed to its role in DNA mismatch repair \parencite{Bernstein2011}.
The repair and complementation hypothesis proposes that sexual reproduction is an adaptive response to incorrect DNA replication, through mutation and damage to the DNA molecule \parencite{Bernstein1984,Bernstein1985,Bernstein1987}.
Recombination repair is the only mechanism currently known which removes double stranded damages to the DNA molecule and such double strand damage is common and could be lethal if not repaired: in human cells such damage occurs approximately 50 times per cell cycle \parencite{Vilenchik2003}.

Recombination and sexual reproduction also plays a role in eliminating detrimental variation from the population, which otherwise would accumulate over time and decrease the fitness of the population (Muller's ratchet) \parencite{Muller1932}.
Recombination produces individuals containing fewer deleterious mutants, helping to reverse the decline in fitness.

The Red Queen Hypothesis also offers an explanation as to why sex has repeatedly evolved in all life forms \parencite{Paterson2010}.
It states that in a rapidly changing environment, alleles that were previously neutral or deleterious and the rapid change makes sexual reproduction advantageous. 
Such rapid changes are proposed to be particularly evident during co-evolution between a parasite and its host \parencite{Decaestecker2007}.

However, despite the advantages of sex, evidence of ancient asexuality has been identified in the genomes of some organisms including root-knot nematodes and bdelloid rotifers \parencite{Lunt2008,MarkWelch2000,Meselson2007a,Pouchkina-Stantcheva2007}.
The classic hallmarks of ancient asexuality are diverged alleles and a lack of phylogenetic incongruence caused by recombination \parencite{Schurko2009}.


\subsection{\textit{Fragilariopsis cylindrus} and Diatoms}

\textit{Fragilariopsis cylindrus} is a species of Diatom: microscopic eukaryotic phytoplanktons, which are found throughout all the worlds’ oceans wherever there is sufficient light and nutrients to support them \parencite{Armbrust2009}.
Diatoms are so named because of their shape and method of reproduction: Their cells are covered by a silica cell wall made of two halves, and they reproduce by asexual mitotic division, decreasing in size each time.
Diatoms occasionally reproduce by forming an auxospore, which reverses the decline in size resulting from reproduction by mitotic division \parencite{Armbrust2009}.
Auxospores also play a role in sexual reproduction, forming after haploid gametes fuse to form a diploid zygote.
Diatoms are an important group of organisms of study because of their role in the ecosystem and in marine biogeochemical cycles \parencite{Assmy2013,Thomas2002,Pondaven2000}.

Diatoms provide an important ecosystem service by performing photosynthesis.
It has been estimated that of all photosynthesis that occurs on earth, one fifth is performed by Diatom species.
Each year diatoms generate as much organic carbon as that produced in total by all the terrestrial rainforests on Earth \parencite{Armbrust2009}.
The organic carbon that is produced by diatoms by photosynthesis is input into food webs: in coastal regions diatoms support fisheries (such as anchovies in the Peruvian ocean) and in the open-ocean, much of the organic matter produced sinks and becomes food for deep-sea organisms (unless is reaches the ocean floor, where it may become sequestered in sediment and rock) \parencite{Armbrust2009,Bowler2010}.
As a result, a significant amount of petroleum deposits under the ocean floor are derived from diatoms sinking.

As Diatoms are found throughout all the worlds’ oceans, they populate interesting and dynamic environments in which environmental factors change and can become extreme.
They are known to be adapted to limited iron, extremes in temperature \parencite{Arrigo2012,Bayer-Giraldi2011,Bowler2010}, salinity \parencite{Krell2006}, and temporal variation in the environment: seasons cause rises and falls in temperature, and freezing and melting sea ice also means the environment’s structure can be heterogeneous through time.
All these extremes occur in the environment of \textit{Fragilariopsis cylindrus}, which is particularly successful in the Southern Ocean, and is often found to form large populations in the bottom layer of sea ice and the wider sea-ice zone including open waters \parencite{Kang1992}.
Such ice is characterized by temperatures below the freezing point of sea water, high salinity caused by the semi-enclosed pores within the ice, and low diffusion rates of dissolved gases and inorganic nutrients \parencite{Thomas2002}.
The environment is not limited in dissolved iron however, unlike the surface ocean \parencite{Wang2014}.
Furthermore, the environment is dynamic: every winter, phytoplankton in the Southern Ocean get locked into sea ice and are released again in the following summer, when most of the sea ice melts \parencite{Vancoppenolle2013}.
However, only a subset of these phytoplankton them have evolved adaptations to cope with this dramatic environmental change, including \textit{F. cylindrus}, which is known to thrive in both habitats \parencite{Bayer-Giraldi2011,Vancoppenolle2013}.

How Diatoms have adapted to such conditions, and become so successful in the oceans, is of interest to evolutionary biologists and genome sequencing has provided insight.
Complete genome sequences are available for two Diatom species (\textit{Thalassiosira pseudonana} and \textit{Phaeodactylum tricornutum}), containing between 10 and 14 thousand genes.
However, of those genes only approximately half can be assigned a putative function based on experimental knowledge \parencite{Bowler2010}.
Furthermore, approximately 35\% of the genes found are specific to each Diatom, which suggests some of them encode adaptations to specific environmental conditions \parencite{Bowler2010}.
As secondary genome sequences became available, the origin of Diatoms seems to be a secondary endosymbiosis between red algae and a heterotrophic eukaryote, and surprisingly many bacterial genes were identified, highlighting the role of HGT in the evolution of Diatom species \parencite{Bowler2010,Raymond2012}.

Diatom specific genes were found to have high diversification rates, and since \textit{Thalassiosira pseudonana} and \textit{Phaeodactylum tricornutum} diverged approximately 90 million years ago, and the two have diverged as much as metazoans had diverged in approximately 550 million years \parencite{Bowler2010}.
It is thought that diversification in Diatoms has been driven by transposable elements, which increased the rate of insertion, deletion, and recombination events \parencite{Bowler2010}.
In contrast, diversification of genes in metazoan genomes during the aforementioned 550 million years, is thought to have occurred largely through whole and segmental gene duplication events \parencite{Bowler2010}.
Some of the diatom specific transposons are activated in response to stresses such as Nitrogen starvation, suggesting diversification of Diatom genes may be stimulated by environmental cues \parencite{Bowler2010}.
The resulting “mix-and-match genomes” \parencite{Armbrust2009} of Diatom species has brought together unique combinations of genes facilitating adaptation to a range of environments, including that encode unique pathways of nutrient assimilation.
comparing the genome of a psychrophile such as \textit{F. cylindrus} with that of diatoms evolved in temperate oceans provides an opportunity to obtain first insights into how this species has evolved to conditions of Southern Ocean waters, and managed to persist for millions of years, underpinning the ecology of an unique food web.

Recently the first large-scale genomic sequencing of \textit{Fragilariopsis cylindrus}, a eukaryotic psychrophilic organism of ecological importance, including whole-genome sequence, transcriptome and population genetic analyses, was completed.
In this thesis chapter I present my contribution to the population genetic analyses of this large body of collaborative work.
This goal of the work described in this chapter was conducted in order to evaluate hypotheses about the evolutionary history of \textit{Fragilariopsis cylindrus}.
These hypotheses were proposed during the genome project, to explain observations about the genome data, and the hypotheses that I tested in this project.


\subsection{The \textit{Fragilariopsis cylindrus} genome project}

The draft of the \textit{F. cylindrus} genome was approximately 60Mb in length,
which is larger than the sequences for the nuclear, plastid and mitochondrial
genomes of the cosmopolitan diatom \textit{T. pseudonana} (34Mb), and the
whole-genome sequence of \textit{P. tricornutum} (27Mb)
\parencite{Armbrust2009,OurFCPaper}.
The draft genome of \textit{F. cylindrus} is smaller in size compared to the
toxigenic coastal species \textit{Pseudo-nitzschia multiseries} (300 Mb)
\parencite{Armbrust2009}.

Assembler programs typically use single end or paired end reads to find overlaps in sequence fragments, joining them to form contigs.
Since it is known that paired end reads are generated from the same DNA fragment, this can help link contigs onto scaffolds, which are ordered assemblies of contigs, with gaps in between them \parencite{Baker2012}.
However, assemblers are not always accurate: one common problem is that if one suspects that the read depth for an assembled region is too high, then it may be that the assembler has merged multiple regions because of their high sequence similarity (typically these are repeat rich regions or duplications) \parencite{Baker2012}.
A second problem is if one suspects that regions of an assembly have a lower read depth than the rest of the assembly, then it may be that those regions represent single polymorphic loci, which have been assembled as two distinct loci \parencite{Baker2012}.
30.2Mb of the scaffolds of \textit{F. cylindrus} could not be collapsed into a single haplotype, because they had greater than 1.5\% nucleotide discrepancies.
The genome contains just over 20,000 protein-encoding genes, and of those, 28\% of them represent alleles that could not be collapsed \parencite{Mock2017}.
The genome contains 46 Mb of collapsed haplotype and 15.1 Mb of diverged haplotype that represents the diverged alleles of the same genetic loci.

The genome contains 21,066 predicted protein-encoding genes, 6,071 genes were represented by diverged alleles, and each pair of diverged alleles had both coding and non-coding regions, and were up to 6\% polymorphic in the non-coding regions.
Comparison of the diverged allele, and non-diverged allele gene ontologies (GO) revealed that genes in the categories ‘catalytic activity’ (GO:0003824), ‘transporter activity’ (GO:0005215), ‘metabolic process’ (GO:0008152), ‘transport’ (GO:0006810) and ‘integral to membrane’ (GO:0016021) were significantly enriched in the diverged alleles set \parencite{Mock2017}.
Furthermore, biological process GO categories ‘metabolic process’ (summarising ‘lipid-catabolic process’ (GO:0016042), ‘glucose metabolic process’ (GO:0006006), ‘oxidation-reduction process’ (GO:0055114) and ‘translation’ (GO:0006412)) as well as GO category ‘transport’-related categories ‘protein transport’ (GO:0015031) and ‘proton transport’ (GO:0015992) enriched in metatranscriptome sequences from Southern Ocean sea ice, and these sequences had high similarity to sequences contained in the diverged alleles of \textit{F. cylindrus} according to BLASTX analyses \parencite{Mock2017}.

Differential expression experiments and RNA-Sequencing suggested that 40\% of
the non-collapsed, diverged allelic pairs showed a 4 fold unequal bi-allelic
expression \parencite{Mock2017}.
This suggested an allele-based adaptation to different environmental conditions.
The differential expression in alleles suggested they were controlled by
separate regulatory systems.
Alleles showing the strong unequal bi-allelic expression were found to have an
elevated rate of non-synonymous mutations, which suggests significant positive
/ adaptive selection and evolution of these allelic pairs \parencite{Mock2017}.
It was concluded therefore, that positive selection has been a driving force in
the evolution of these alleles and hence the adaptation of this diatom to the
environmental conditions it faces.

An evolutionary explanation of the 28\% of genes that could not be collapsed
(i.e. diverged genes) is desired, as it would explain one of the mechanisms
through which this diatom appears to have adapted to its polar environment.
However, this signature of positive selection alone does not provide a
sufficient evolutionary explanation: Meiotic recombination, which occurs during
sexual reproduction, should act to homogenize any two alleles of one gene in the
diatom genome.

Allelic divergence is a classic signature in genomes of organisms called
“ancient asexuals” \parencite{Little1996,Pouchkina-Stantcheva2007,Schurko2009}.
By its definition asexuality is a negative proposition, based on an apparent
lack of sexual reproduction in an organism, and since absence of evidence is
not equivalent to evidence of absence, ancient asexuality is a difficult
proposition to demonstrate in an organism absolutely
\parencite{Schurko2009}.
Indeed the existence of ancient asexuals has been debated and doubted in the
past \parencite{Judson1996,Little1996}, and this is perhaps
unsurprising considering current theory explaining the benefits of, and
maintenance of sexual reproduction.

If the divergence of alleles is due to ancient asexual reproduction, then the
recombination rate between these alleles should be reduced.
It was also expected that phylogenetic networks would have a very clear
structure, with deep branches. To test these predictions and evaluate empirical
data I performed population genetic simulations. More detail is presented in the methods section, but briefly, sequence data was available to
test for the evidence of recombination based on an environmental sample of
\textit{F. cylindrus}, that was amplified by PCR and sequenced using Sanger sequencing.
It resulted in 200 high quality sequences from alleles of Ferrichrome ABC
transporter and Large Ribosomal Protein L10, and the signature of recombination
between these alleles was analyzed as well as several other population genetic
parameters.

This project had the aim of establishing whether ancient asexuality and a lack of recombination is evident, by establish whether recombination has occurred by analyzing the aforementioned DNA sequences.

The specific aims were:
\begin{itemize}
    \item Use LAMARC to establish a population recombination rate and population
    Theta parameter.
    \item Use the incompatible sites test to detect evidence of phylogenetic
    incompatibility (and therefore recombination) between closely related
    sequences.
    \item Visualize recombination signal of choice sequences with the
    HybridCheck package.
    \item Conduct a comparative phylogenetic network analysis.
    \begin{itemize}
        \item Construct un-rooted phylogenetic networks of alleles present
        in the natural sea-ice populations.
        \item Construct un-rooted phylogenetic networks from \textit{silico}
        populations simulated using simuPOP.
        Some of these \textit{silico} populations were simulated under asexual
        (clonal) regimes of reproduction, and some were simulated under a
        sexual reproduction regime, with different mutation and
        recombination rates.
        \item Compare the empirical networks with those simulated, to try
        and suggest the mutation and recombination rates the Diatom
        population may have in nature.
    \end{itemize}
\end{itemize}
