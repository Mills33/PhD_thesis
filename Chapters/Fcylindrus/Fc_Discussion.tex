\section{Discussion}

The phylogenetic networks resulting from population genetic simulations support several assumptions we had about how recombination, and population mutation rate ($\theta$) may be inferred from phylogenetic networks.
Specifically, (1) the levels of Theta affect the average branch lengths of the networks, and (2) the extent of recombination affects the number of splits in phylogenetic networks.
These two assumptions are not controversial: a higher population mutation rate leads to more mutations in a population the same amount of time, and thus would lead to longer branches in any phylogeny or network computed for sequences sampled from the population \parencite{Frankham1996,Hein2004,Wakeley2008}.
Phylogenetic Split Networks \parencite{Huson1998} were conceived of as a way to detect and represent reticulate evolution.
Wherever there is a non-tree like structure or “loops”, recombination may be inferred.
The networks resulting from the simulations confirm these assumptions, and so give confidence in any inferences made about the population and evolution of \textit{F. cylindrus} from the networks of the ABC Iron Transporter sequences, and the Large Ribosomal Subunit Sequences.

Secondly, from comparisons between the networks of the ABC Iron Transporter sequences, Large Ribosomal Subunit Sequences, and simulated networks, it was concluded that LAMARC \parencite{Kuhner2006a} estimate of $\Theta$ was a reasonable estimate for the population of \textit{F. cylindrus}.
It was also concluded that these networks provide evidence of recombination for the sequences of the ABC Iron Transporter, and in the sequences of the Large Ribosomal Subunit.
Evidence of recombination does not have to mean that an organism is reproducing sexually, meiotic recombination is associated with sexual reproduction, but mitotic recombination could also explain the recombination signal detected in these sequences.
However, whilst mitotic or meiotic recombination may explain the recombination signal in the sequences, it was concluded that ancient asexuality is not a likely explanation, because of the lack of similarity of the ABC Iron Transporter, and the Large Ribosomal Subunit networks, to the networks generated by simulations of ancient asexual evolution.


\subsection{Sex and the diatom reproductive cycle}

Even though sexual reproduction has not been observed in the lab cultures of \textit{F. cylindrus}, this diatom does not appear to be an ancient asexual.
This might not be surprising given what is already known about Diatom biology and sexual reproduction.
The typical cell cycle of Diatoms is diplontic i.e. the vegetative cells are diploid, and the haploid gametes are short lived \parencite{Chepurnov2004}.

The Diatom life cycle features two key phases which may be summarized by the following:

The first phase is a long vegetative phase; this phase can last for months or years. During this phase, vegetative cells divide by mitosis, gradually becoming smaller. 
The cell size decrease during the vegetative phase of the diatom life cycle is due to the shape and structure of Diatom cell walls and the division pattern of the Diatoms. 
The cell wall is made of sillicated components, which together are termed the frustule. The frustule is made of two overlapping halves or thecae \parencite{Chepurnov2004,Davidovich1998,Poulickova2008}.
These thecae are not the same size, the larger of the two thecae is called the epitheca, and the smaller of the two is called the hypotheca.
When mitosis occurs, cytokinesis splits the diatom where the two thecae overlap.
The two resulting daughter cells inherit one of the parent cell’s two thecae as its own epitheca, and they grow their own hypotheca \parencite{Chepurnov2004}.
Since one of the daughter cells inherits a hypotheca as it epitheca, it will be smaller in size to its parent cell.
Thus the average cell size of a population of diatoms decreases as mitotic cell division occurs.

The second phase is shorter, and includes sexual reproduction and the production of new vegetative cells, restoring the cell size \parencite{Chepurnov2004}.
Production of gametes during the sexual reproduction phase has been demonstrated to occur by classical meiosis in many Diatom species.
Diatoms restore their cell size through the production of auxospores, which result from sexual reproduction \parencite{Davidovich1998}.
During auxosporulation, recombination and cell size restitution occurs: gametes fuse to form the auxospore, which expands and a new cell is produced within.
The cell walls of the gamete producing cells are lost, and so the auxospore must then form the shape of the vegetative cells de novo \parencite{Chepurnov2004}.
If a population of Diatom cells did not undergo sexual reproduction to produce the auxospores to restore their cell size, the population would gradually decrease in cell size until they become critically small.
At this point the population would die, and this has been observed in experimental cultures.
Diatom cells can only become sexualized when they are sufficiently small, but they may also not be able to become sexualized if they become too small or hit the critical cell size before they die \parencite{Chepurnov2004,Davidovich1998,Poulickova2008}.
The maximum size of initial diatom cells, the maximum and minimum sizes of cells capable of sexual reproduction, and the minimum size before death are strict for each diatom species and are termed cardinal points \parencite{Chepurnov2004}.

However, despite the role of sex in the restoration of cell size in diatoms, it is not always necessary for cell size restoration.
For some diatom species, asexual auxosporulation is a possibility, presumably it is some secondary modification of a developmental pathway that was sexual, and some species do not even undergo auxosporulation and exist as entirely as asexual populations, and their cell size is restored by vegetative cell enlargement \parencite{Chepurnov2004,Gallagher1983,Nagai1995,Sabbe2004,werner1977biology}.
Species such as \textit{Caloneis amphisbaena} and \textit{Sellaphora pupula} "lanceolate" have been found to exist in populations of a very limited range of cell size, and this cell size has remained unchanged after many generations of observation \parencite{Mann1989,Mann2004}.

Therefore, whilst sex is a common feature of the diatom life cycle, and is important for cell size restoration in many species, it is not unreasonable to suggest the hypothesis that a diatom like \textit{F. cylindrus} could have evolved asexually for a long period of time.
However, the network reconstructions and evidence of recombination demonstrated by this study cast doubt on that hypothesis as an explanation for the diverged alleles.


\subsubsection{Allelic Divergence in diatoms may be explained by population size}

If the “ancient asexuality hypothesis” is rejected as the explanation of the diverged alleles in \textit{F. cylindrus}, then an alternative explanation of how this diatom evolved diverged and functionally differentiated alleles is desired.
These alleles show signatures of positive selection, and they are differentially expressed.
The question is; assuming sexual reproduction and recombination, why does recombination not homogenize the sequence variation between two alleles over time? 

An alternative hypothesis explaining the adaptive evolution of \textit{F. cylindrus} is a large population size, which would lead to bigger coalescence times between maternal and paternal loci.
In combination with a low recombination rate, this would result in independent adaptive evolution and divergence of the different haplotypes.
This is intuitive if one considers a coalescent process back through time of an idealized population, because the coalescent relates genetic diversity to demographic history.
In such a process, the probability that any two lineages extant at time $t$, coalesce in the previous generation $t – 1$, is the probability that they share a parental DNA sequence.
For a diploid population there are $2N_e$ alleles in every generation, assuming a constant population size \parencite{Hein2004}.
Assuming random mating and neutral evolution, the probability any two alleles coalesce in the previous generation (i.e. they share the same parental sequence) is $1 / (2N_e)$.
Therefore, the probability those two alleles do not coalesce, is $1 – (1/(2N_e))$.
These probabilities are dependent on the size of the population in question \parencite{Wakeley2008}.
Larger populations, result in a smaller probability that two alleles coalesce in the previous generation, and a greater probability that they do not.
With each successive previous generation, the probability of coalescence is geometrically distributed \parencite{Hein2004,Wakeley2008}.
This means that it is the product of coalescence at the generation of interest and the probability of non-coalescence at the preceding generations i.e.
\begin{equation}
P_c(t)=\bigg(1-\frac{1}{2N_e}\bigg)^{t-1}\bigg(\frac{1}{2N_e}\bigg)
\end{equation}
From this equation, it can be seen that with larger populations, the probability that two alleles coalesce further back in time is greater i.e. the expected coalescence time between two alleles is larger, therefore alleles are expected to be more diverged.

This explanation is consistent with the estimation of a $\Theta$ of 0.066 by the LAMARC \parencite{Kuhner2006a} analysis, which is also supported by the simulations.
The population mutation rate $\Theta$, is proportional to the product of the mutation rate and the effective population size and so the value predicted by LAMARC could be the result of a very large population.
Furthermore, prior research has been performed to estimate the abundance of \textit{F. cylindrus} in water columns around the Antarctic \parencite{Kang1992}.
During the summer, numbers of $7.9 \times 10^{10}$ cells $^{m-2}$ were observed, and during the winter, numbers of $1.1 \times 10^8$ cells $^{m-2}$ were observed.
Marginal ice zones are known to be sites with much dynamic activity such as jets, eddies, currents, melting, freezing, and upwelling \parencite{Kang1992}.
They are also known to be sites of increased phytoplankton biomass and primary productivity, due to their light levels, ice-distribution, and vertical stability.

Therefore, the hypothesis that a large population size explains the levels of diversity is consistent with both population genetic (coalescent) theory, results of this study, as well as the findings of other research.
It is also attractive, because of its simplicity.
It is much more plausible that a phytoplankton species has very large populations; than it is that the species had abandoned sex as a reproductive strategy:
Sex is a common aspect of the diatom life cycle and is often essential for cell size restoration and population survival.
Furthermore, as was explored in the Introduction, there is a substantial body of theory explaining why sexual reproduction evolved two become a widespread reproductive strategy, and is advantageous, despite the apparent costs.


\subsubsection{Study limitations and subsequent FALCON assembly}
However, this study has some limitations which should be acknowledged when considering these results.
First, whilst evidence of recombination in the form of the splits networks and the presence of incompatible sites is obtained from these sequences, it was not possible to examine any larger recombination events or blocks as was possible for \textit{Albugo candida} in chapter \ref{chap:Acandida}.
Indeed, the number of informative sites was too small for the HybridCheck software (which was implemented to analyse large contigs) to effectively run.
Secondly, the analyses were only performed on two genes.
Whilst it was concluded that the two genes were representative of the larger set of diverged alleles (figure \ref{fig:FC_Res_1}), we do not know if the PCR primers amplified only maternal and paternal alleles of those genes or if some of the sequences amplified also represent paralogues.

The question of whether the diverged alleles observed in the assembly were truly diverged alleles was resolved for the assembly described in the introduction experimentally.
Single haplotyped fosmids were Sanger sequenced by collaborators, providing contiguity information and they were compared with the assembled genomic scaffolds, and an annotated protein set from the diverged regions in the genome.
Data from these comparisons revealed a clear separation between allelic pairs and gene duplications based on 100\% identity to the haplotyped Sanger sequenced fosmids.
Additionally, the nucleotide similarity of the diverged alleles (mean = $97.01 \pm 0.03\%$) is significantly (p-value $< 10^{-09}$) higher than for gene duplicates (mean = $84.07 \pm 0.36\%$). 
Therefore, whilst it may be that some uncollapsed regions of the assembly could be duplicates, there is high confidence that the allelic pairs identified are indeed diverged alleles and not duplicates.

Since this work has been completed, an assembly has been completed using PacBio long read sequencing technology, which has also supported that true diverged alleles have been identified and that they are not duplicated sequences (although duplicated sequences are indeed present in the genome). 
The sequencing work and library preparation was completed by the platforms and pipelines team.
A 20kb fragment length library was constructed, and a 4kb insert size library was also created. 
Both libraries were sequences using the PacBio RS2 instrument, using SMRT cells with the c4P6 chemistry.
The 20kb fragment length library yielded 1.37Gb of data, and the 4kb insert size library yielded 3.85Gb of data.
The final N50 of read length varied between 8215 to 8898bp for the 20kb fragment length library, and the N50 ranged from from 2558 to 2680bp for the 4kb insert size library.

Assembly was completed by collaborator Pirita Paajanen, who combined the data from the SMRT cells and filtered the shortest reads, yielding 3.8Gb of data which gave 63x coverage.
Assembly was completed using the diploid aware PacBio assembler, Falcon 0.3.0.
The output of the Falcon assembler was divided into two parts.
The haploid assembly resulted in primary contigs from which a genome size of 59.7Mb was deduced. 
However, the assembler also produced alternate contigs which were the result of the assembler being unable to decide between two possible routes through the genome graph the genome.
Such 'bubbles' in the genome graph represent diverged haplotypes, containing diverged alleles.

The haplotype divergence differed between chromosomes: The longest chromosome 000000F had only one alternate contig with a length of 6047bp.
In contrast, contig 000002F was 1246645bp long and had 14 associated alternative contigs, of a total sequence length of 633764bp.
For each of the 14 alternative contigs of chromosome 000002F, I extracted and aligned the two haplotype sequences using the pairwise alignment algorithm available in the Bio.jl software package (https://biojulia.github.io/Bio.jl), using an EDNA scoring matrix.
Once aligned, a non-overlapping sliding window was moved across the sequences, and the p-distance between the sequences within each window was calculated. For each computation, the width of the sliding window was set as 1\% of the width of the pairwise alignment in (bp).
The results of this analysis are included as extra information in the appendix, figure \ref{fig:falconwindows}.
The figure demonstrates different levels of divergence across the diverged haplotype pairs, including the appearance of indels between some haplotype pairs.
Further work will show how the sequences of allelic pairs align to the FALCON assembly, revealing which pairs align to different haplotypes of a FALCON 'bubble' (true allelic pairs), and which pairs align to the same haplotype of a FALCON 'bubble' (potentially gene duplicates).

At the time of writing, multiple population samples of \textit{F. cylindrus} are not available, and so analyses presented here used sequences from cultures, and so further population genetic analyses should be conducted in the future as more data becomes available, for example to assess the population structure of \textit{F. cylindrus} and investigate if gene flow is occurring between subpopulations of \textit{F. cylindrus}.

The fact that the genome assembly contains some duplicates, and that some of the allelic pairs analysed in this study may be found to be duplicates is not problematic for the hypothesis that this Diatom species has adapted through alleleic divergence, as it may be argued allelic divergence could lead to gene duplication and the conditions for the divergence of alleles and the divergence of duplicates overlap: 
When diverged alleles are maintained in a population due to heterozygote advantage, duplications may rapidly spread through the population, causing an individual to act as a genetic heterozygote yet still breed true.
\cite{Proulx2006} argued that genetic redundancy is the mechanism usually cited as allowing duplicate genes to diverge, but redundancy is present in a diploid before duplication: Dominance creates the same kind of redundancy duplicates have, but for alleles of single copy genes.
Therefore mode of inheritance is the thing then which most distinguishes duplicates from single copy genes: Segregation prevents the fixed inheritance of alternative allelic variants at a single locus \parencite{Proulx2006}.
In other words, heterozygotes at one locus are broken up by segregation during sexual reproduction, whereas duplicate loci in an individual can carry copies of alternate alleles at different loci.
Their results show that fitness relationships that allow divergent alleles to evolve at one locus overlap significantly with those that allow the divergence of previously duplicated genes at two different loci \parencite{Proulx2006}.


\subsubsection{Conclusion}

The genome of the polar diatom \textit{Fragilariopsis cylindrus} contains diverged alleles that are differentially expressed in different environmental conditions.
Evidence of recombination was found which contradicts the ancient asexuality hypothesis explaining how these diverged alleles may have evolved.
An alternative, competing hypothesis is proposed, supported by the evidence presented, that a large population size has allowed diversifying selection to differentiate the alleles of genes despite the homogenizing effect of recombination.
Additional population samples, and analysis of larger contigs made possible by improved genome assembly for recombination, will help answer the question of how \textit{F. cylindrus} has evolved this remarkable strategy to cope with varying environmental conditions.



