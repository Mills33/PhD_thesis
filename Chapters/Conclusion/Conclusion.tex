\chapter{General Conclusion}

\subsection{Summary and Conclusions}

In this thesis, work focused on how recombination facilitates the adaptive evolution of a plant pathogen and a polar marine diatom.
Both of these organisms were of evolutionary interest due to aspects of their lifestyles and/or physiology:
The plant pathogen \textit{Albugo candida} was of interest because whilst it was an obligate biotroph, it has a very large host range, and the diatom \textit{Fragilariopsis cylindrus} was of interest because the genome sequencing project and differential expression experiments revealed genes with diverged alleles that were differentially expressed in different environmental conditions.

Recombination is important for the formation of novel genotypes, haplotypes and alleles, therefore is plays a key role in adaptive evolution \parencite{Grauer2000}.
Recombination separates deleterious mutations from their genomic background, in combination with purifying selection this reduces the mutational load \parencite{Lynch1990a}.
Recombination also brings beneficial mutations from separate lineages into one individual or lineage.
However, recombination also plays a fundamental role in the repair of damaged DNA, when homologous recombination replaces a damaged DNA strand with its intact counterpart, and it was likely this function of recombination that was important in early prokaryotic life and evolution \parencite{Cavalier-Smith2002}.
With respect to adaptive evolution, however, the principal consequence of recombination is that it generates novel combinations of nucleotides, which in turns allows for selection to act a much finer scale, i.e. at the level of nucleotides rather than the entire genome.

The potential of recombination to generate novel allelic combinations is important for host and pathogens which are engaged in an evolutionary arms race to adapt and counter adapt to each others molecular mechanisms of pathogenicity or immunity.
The red queen hypothesis explains the advantage of sexual reproduction in such terms.
The variability generated by sexual reproduction (and meiotic recombination) results in genetically unique offspring, which permits a faster response to selection \parencite{Paterson2010}.
As a result sexually reproducing species are able to improve their genotype in changing conditions. 
Co-evolutionary interactions between host and parasite select for sexual reproduction in hosts in order to reduce the risk of infection.
Oscillations in genotype frequencies are observed between parasites and hosts in an antagonistic co-evolutionary way without necessitating changes to the phenotype, and in host-parasite co-evolution systems with multiple hosts, Red Queen dynamics may affect which host and parasite types become common (or rare) \parencite{Charlesworth2010}.

It was hypothesized that the \textit{Albugo candida} species was composed of several host-specialised races, each locked in an evolutionary arms race with their specific host.
Such a race with a specific host would lead to further divergence and possibly speciation of the races.
However, \textit{Albugo} is known to be able to suppress non-host resistance.
Infections of \textit{Albugo sp.} could suppress the runaway cell death phenotypes of plants, allowing formerly avirulent strains of downy mildew to infect \parencite{Cooper2008}.
Assuming that this ability extended to other non-host species, \textit{Albugo} may be modeled as a 'microbial hub': taxa that are integral and highly connected to the network of a hosts microbial community.
Such hubs may affect community compositions through microbe-microbe interactions or, as seems to be the case with \textit{Albugo}, suppression of host defense responses \parencite{Agler2016}.
Therefore, non-host immune suppression would enable host-specific races of \textit{Albugo candida} to overcome the ever increasing barrier to gene flow that specialisation imposes, and sexual reproduction between races, followed by introgression by back-crossing, would permit the generation of a range of novel genotypes.
Consequently the species could evolve its wide host range.

To assess this hypothesis it was necessary to scan the genome of \textit{Albugo candida} isolates to identify recombinant regions.
Furthermore, to distinguish such regions as recombinant and not the result of incomplete lineage sorting due to rapid divergence, the regions identified needed to be tested for significance and the coalescence times estimated.

Scans of the genomes for recombination revealed a highly recombined mosaic genome, and therefore a rapid coalescence estimation method for all of the recombination blocks was desired, in addition to a method of plotting which effectively demonstrated the high degree of mosaic-ism in the \textit{A. candida} genome.
Therefore, rapid detection and dating of recombination blocks was implemented, and the software package HybridCheck was created  and tested using simulated data as in chapter \ref{chap:HC}.
HybridCheck was also tested for consistency with RDP analyses of \textit{A. candida}, which identified recombination, and BEAST estimates of coalescence times for a subset of the identified recombination regions (chapter \ref{chap:Acandida}).
The evidence presented in chapter \ref{chap:Acandida} confirmed the model of \textit{Albugo candida evolution}:
Isolation, divergence and specialisation of races generates repertoires of effectors for a specific race.
Those adapted repertoires are then brought together when two races hybridize.
The result if the generation of novel repertoires of novel combinations of these effectors.
Specific avirulence effectors that trigger host immunity may be lost through segregation and through loss of heterozygosity \parencite{Lamour2012PathogenCapsici,McMullan2015a}.
Hybrids, with new combinations of effectors, and having lost effectors which impeded their colonisation of other hosts previously, may expand their geographical range and population size clonally.
Some of these hybrids may be able to colonise new hosts, expanding the host range.

The genome assembly project of \textit{F. cylindrus} revealed that the genome contained 21,066 predicted protein-encoding genes, 6,071 genes were represented by diverged alleles, and each pair of diverged alleles had both coding and non-coding regions, and were up to 6\% polymorphic in the non-coding regions.
Furthermore, differential expression experiments and RNA-Sequencing suggested that 40\% of
the non-collapsed, diverged allelic pairs showed a 4 fold unequal bi-allelic
expression \parencite{Mock2017}.

Alleles showing the strong unequal bi-allelic expression were found to have an
elevated rate of non-synonymous mutations, which suggests significant positive
/ adaptive selection and evolution of these allelic pairs \parencite{Mock2017}.
It was concluded therefore, that positive selection has been a driving force in
the evolution of these diverged alleles and hence the adaptation of this diatom to the
environmental conditions it faces.

An evolutionary explanation was hypothesized: The alleles of an allelic pair could diverge as a result of positive selection because there was a long history of asexual reproduction in the organism, and hence an absence of recombination acting as a homogenizing force between alleles.

However, results from recombination detection analysis, and phylogenetic network construction of PCR amplified sequences from DNA extracted from \textit{F. cylindrus} cultures conflicted with results of the same analyses performed with DNA sequences obtained by population genetics individual based simulations of ancient asexuality.
Indeed the results for \textit{F. cylindrus} were more consistent with those of simulations of a scenarios of sexual reproduction, and a large $\Theta$ value.
This result suggests an alternative competing hypothesis, that very large effective population sizes could have led to the divergence of the alleles in each allelic pair as a result of positive selection, in the face of the homogenizing influence of recombination through sexual reproduction.


\subsection{Impact and potential future directions}

\subsubsection{\textit{Albugo candida}}

A paper describing the extent of the introgression identified within the \textit{A. candida} genome was published in eLife \parencite{McMullan2015a}.
According to Google Scholar, the study has been cited 11 times at time of writing.
Citations include reviews of the role of hybridisation and introgression in the adaptive evolution and emergence of new fungal and filamentous plant pathogen strains \parencite{Depotter2016,Dong2015,Stukenbrock2016}, research demonstrating the role of recombination in the evolution of the Rp1 resistance genes in grasses \parencite{Jouet2015}, and a study presenting evidence that for \textit{Coleosporium ipomoeae}, any genotypes can infect multiple hosts from non-local communities, but only are highly host specific when tested on hosts from local communities, calling into question theoretical results of single-pathogen single-host studies which suggest that selection favours genotypes with a broad host range \parencite{Chappell2016}.
Following the 2015 eLife paper, \cite{Belhaj2015} published a more extreme example of the ability of \textit{Albugo spp.} to suppress the host immune system.
They found that \textit{Phytophthora infestans}, which is typically a potato and tomato specialist pathogen, was capable of infecting the plant model organism \textit{Arabidopsis thaliana} when \textit{Albugo laibachii} has also colonized the plant.
The nature of the \textit{P. infestans} infection was similar to that of an \textit{Albugo laibachii} infection: Transcription profiling of \textit{P. infestans} infections revealed a significant overlap between the sets of secreted proteins of \textit{P. infestans} during infection of \textit{Arabidopsis thaliana} and during infections of potato.
This suggests there is similar gene expression dynamics on the two species, and it raises the question.
Is gene flow between two different Oomycete species possible? And could this contribute to adaptive evolution of these pathogens.

It is well established that \textit{Albugo} suppresses hon-host immunity in hosts it infects, and as a result of work presented in this thesis it was concluded that this lowers barriers to gene flow and permits introgression, facilitating the generation of novel pathogen haplotypes and enabling \textit{Albugo candida} to evolve a wide host range.
However, this model of \textit{Albugo candida} evolution raised a conceptual problem: 
This phenomenon  appears to extend to other pathogen species that were not \textit{Albugo spp.} \parencite{Belhaj2015}, and therefore \textit{Albugo spp.} may act as a microbial hub as previously noted.
If this is the case, how is it that \textit{Albugo spp.} (obligate biotrophs with a vital dependence on the host) can compete in this limited niche, whilst at the same time enable non-host colonization for other pathogen species who are then presumably competitors for the same resource.
An answer to this problem was provided by a paper from \cite{Ruhe2016}.
Shotgun proteomics was completed of the apoplastic fluid of samples of lab-grown \textit{Arabidopsis thaliana} that were infected with \textit{Albugo spp.}, and samples which were uninfected.
Work was repeated for wild-grown \textit{Arabidopsis thaliana} and they found that whilst both lab-grown and wild-grown \textit{Arabidopsis thaliana} supported extensive \textit{Albugo} colonization \parencite{Ruhe2016}.
However, no or low levels of defense-related proteins were detected in lab samples, but regardless of \textit{Albugo spp.} infection status, wild plants showed a broad spectrum of defense-related proteins at high abundances and lab-grown plants did not.
These results suggest that \textit{Albugo spp.} do not strongly affect immune responses and leave distinct branches of the immune signaling network intact \parencite{Ruhe2016}.
This suggests that the pathogens of the \textit{Albugo} genus, including \textit{Albugo candida} in the wild are fine tuned to avoid triggering strong host defense reactions, but also to avoid a broad-spectrum host defense suppression, thus allowing them to avoid competition from other species growing in the same niche \parencite{Ruhe2016}.
Since races of \textit{Albugo candida} are members of the same species, they may still colonize the same host plant at the same time, allowing introgression to occur (explaining the introgression signal observed), but other more distantly related competing pathogens may be excluded by this precise host immunity manipulation observed by \cite{Ruhe2016}, and so may not get to compete with \textit{Albugo spp.}.
However this experiment only examined \textit{Arabidopsis thaliana} as a host, and crops grown in monoculture are often uniform and subject to artificially maintained conditions and treatments, and this may be considered analogous to plants grown in laboratory conditions.
So it is uncertain whether in monoculture environments \textit{Albugo spp.} manipulate their host immune systems subtlety and precisely, thus avoiding colonization of competition, or whether as with lab-grown \textit{Arabidopsis thaliana} they do significantly affect the secretome of the host, allowing competitors to colonize.

In the future, additional study of more strains and population samples of \textit{Albugo candida} is desirable, since the study presented in this thesis only examined the genomes of three 'races', and more samples might increase the number of \textit{Albugo candida} races we can analyse.
Future potential work also includes disentangling the true branching order of \textit{Albugo candida} races, and improving the detection and dating methods used to analyse \textit{Albugo candida} genomes (see below).


\subsubsection{HybridCheck}

The HybridCheck software package was initially created out of a need specific to the \textit{A. candida} project in chapter \ref{chap:Acandida}.
Following the \textit{A. candida} project, the HybridCheck software was published in a short software note in Molecular Ecology Resources \parencite{Ward2016}, and other groups across the Norwich Research Park became interested in using it with their own study systems.

In particular, researchers at Norwich Medical School working on \textit{Cryptosporidium} used HybridCheck to perform chronological assessment of recombination events identified in the genomes of  
three trains of \textit{C. parvum} (IIaA15G2R1, IIcA5G3j, IIcA5G3a), and a single \textit{C. hominis} (IbA10G2) GP60 sub-type strain \parencite{Nader}.
They found 104 unique recombination events, and a skewed distribution of recombination events across chromosomes.
More recombination events were identified on chromosome 6, and a greater number of events was observed for \textit{C. parvum anthroponosum} sub-type IIcA5G3a than for any other strain.
More than 90\% of all recombination events occurred proximal to loci suspected to drive virulence or play a major role in host-parasite interactions in human cryptosporidiosis.
Therefore it appears that in this pathogen too, recombination is an important force, generating novel gene combinations and driving the adaptive evolution of a pathogen to its host \parencite{Nader}.
The estimated divergence dates calculated in their study provide the first chronological description for genetic introgression between human-infective \textit{Cryptosporidium spp.}.
HybridCheck analyses revealed a chromosome-wide consensus that places a majority of introgression events between zoonotic (IIaA15G2R1 and IIcA5G3j) and anthroponotic (IIcA5G3a) \textit{C. parvum} sub-type strains at approximately 10-15 thousand generations ago, while genetic introgression (or recombination) between the two more closely related zoonotic strains appears to be more recent (between approximately 3 to 5 thousand generations ago) \parencite{Nader}.

Based on infectivity studies in healthy adult volunteers, the average generation time within a host is 14.8 hours, and assuming a steady rate of transmission within host populations, they derived a minimum estimate of the recombination events of around 5.9 (zoonotic vs. zoonotic \textit{C. parvum}), 17.6 (zoonotic vs. anthroponotic \textit{C. parvum}), and 176.7 (\textit{C. hominis} vs. \textit{C. parvum}) years ago \parencite{Nader}.
In other words, they estimate that the evolutionary split between the two primary human-infective species appears to have occurred at the turn of the second industrial revolution, around 1840 \parencite{Nader}.

Whilst this result is putative and needs validation with other dating methods before publication submission, it is a clear demonstration of the utility of HybridCheck for researchers in estimating coalescence times rapidly, across many recombination affected genomic regions.

Future directions for work involving HybridCheck include its continued use in other organisms.
For example HybridCheck is already being used to generate preliminary results for population genomic data for mice (\textit{Mus spp.}), being generated at the Earlham Institute, with the aim of confirming hypotheses of genetic isolation between species, and identifying potential introgressions between populations.
Future work involving HybridCheck may also involve programmatic work.
Bioinformatics methods and the detection of introgression is an active area of research, and more algorithms and methods will likely be created in the future.
Therefore, HybridCheck would have greater utility as a provider of different methods for the detection and dating of recombinant and introgressed regions, that are able to work on multiple different data sources or formats.
As a programming problem, such software code might be best implemented, using multiple dispatch, to make it more easily maintained, and more easily used.
Multiple dispatch is a feature of some programming languages in which a function (sometimes called a  method) can be dynamically dispatched based on the type of more than one of its arguments.
This thesis author has already co-founded, develops, and maintains a new bioinformatics infrastructure and community called BioJulia, based around a modern new programming language for scientists and technical programmers, called Julia.
The language is high-level, implements a flexible type and multiple dispatch system, and can achieve speeds matching those of compiled software written in the C language, with less lines of code.
These features make it ideal for the kind of rapid and flexible development that Bioinformaticians often do, and should development of HybridCheck continue towards this goal, the framework already has many high performance code modules and features that a BioJulia port of HybridCheck could take advantage of.

In the near future, approaches to recombination detection may also change.
Currently, HybridCheck and other methods typically analyze DNA or protein sequences and identify regions that are phylogenetically incongruent i.e. where computed phylogenetic topologies change or there is a change-point in computed genetic distances.
After the identification of these regions, it may be assumed they are recombination, or incomplete lineage sorting, and subsequent analyses, such as the dating method in HybridCheck, may be employed to try to distinguish whether the cause is recombination or incomplete lineage sorting.
The cause may also be assumed based on rates of speciation or population size; incomplete lineage sorting is more likely when either of the two are high.
However, as described in chapter \ref{chap:HC}, there are problems with this approach which leave room for future improvement.

For example, recombination blocks can become fragmented by accumulation of subsequent mutations following the recombination event.
Consequently, older recombination blocks tend to be smaller, when they are actually larger.
Thus, not all mutations are accounted for, resulting in an underestimate of the divergence time particularly for old recombination events/regions of incomplete lineage sorting.

Furthermore, some methods of resolving introgression from incomplete lineage sorting require knowledge of branching orders, and sometimes these are unknown, and sometimes this is even because of the influence of introgression or incomplete lineage sorting.
To solve this issue for the malaria parasite, \cite{Neafsey2014} obtained the correct species branching order of the \textit{An. gambiae} complex and two Pyretophorus out-group species.
To do this in the face of introgression and incomplete lineage sorting they used 50kb non-overlapping windows across a genome alignment and computed phylogeneies for each window.
At least 85 tree topologies were observed.
When these were sorted according to chromosome arm and their relative frequency, the most commonly observed topology for the X chromosome was highly discordant with the most commonly observed topology for the autosomes.
They then grouped these phylogenetic toplogies, into three distinct topology categories based on the relative phylogenetic positions of two species: \textit{An. arabiensis} and \textit{An. quadriannulatus}, and they observed the topology category most commonly observed on the X chromosome, was not the same as for the autosomes. 
Dating the internal nodes of phylogeneies for each topology category allowed them to distinguish which category of topology best represented the true branching order, and which represented topologies that were caused by introgression.
Given that almost all of the autosome was represented by a topology category that is affected by introgression and linkage disequilibrium, traditional phylogenetic methods for resolving a species level topology, which typically invoke some majority rule, would certainly have resulted in the incorrect answer.

The method utilized in their work will be of great benefit to researchers studying complicated genomes where introgression, and incomplete lineage sorting, are prevalent.
A likely future direction for the development of HybridCheck will be to take these methodological ideas and implement tools that make it trivial for researchers to decompose the gene trees computed across a genome, identify topological categories from those trees, and organize them, before analyzing the divergence times of the phylogenies in each topological category.
In the future HybridCheck should make it simple to perform such an analysis along with other methods such as Patterson's D, $f_d$, and tests to distinguish introgression from incomplete lineage sorting.
It should make it trivial to compile such multiple lines of evidence into a more complete picture of introgression, incomplete lineage sorting, and linkage, across genomes.


\subsubsection{\textit{F. cylindrus}}

The study of \textit{F. cylindrus} is in preparation to be submitted to the journal Nature this year.
As such it is not possible to describe the impact in terms of a number of citations, or who has cited it and why at this time.
However, as stated in discussion of chapter \ref{chap:diatom}, reviewer comments led to further sequencing with PacBio technology, which resulted in confirmation that we had obtained strong evidence of diverged alleles.
Furthermore, it is known that at time of writing, that unpublished data and correspondence from a colleague and co-author of the paper, Chris Bowler (perscom), that similar evidence of diverged alleles and differential expression has been found in another diatom species that his group study.
Therefore, it could be that the data presented in this thesis and in the paper, are the first evidence of a common phenomenon and mechanism of adaptation in this group of organisms.
Future work on this topic has already been described in the discussion of chapter \ref{chap:diatom}:
Imminent future work will show how the sequences of allelic pairs previously identified align to the new FALCON assembly.
This will reveal which pairs align to different haplotypes of a FALCON 'bubble' (true allelic pairs), and which pairs align to the same haplotype of a FALCON 'bubble' (potentially gene duplicates).
Currently, multiple population samples of \textit{F. cylindrus} are not available, and so analyses presented here used sequences from cultures, and so further population genetic analyses should be conducted in the future as more data becomes available, for example to assess the population structure of \textit{F. cylindrus} and investigate if gene flow is occurring between subpopulations of \textit{F. cylindrus}.

In conclusion, detecting and understanding how recombination is affecting the genomes is critical to understanding how species of interest evolve and adapt to dynamic environments, this thesis has demonstrated how recombination appears to have influenced the evolution and adaptation of two different eukaryotic micro-organisms.
Future work will expand on the bioinformatics methodological techniques implemented in this thesis, as more and more data becomes available for these two species.  


