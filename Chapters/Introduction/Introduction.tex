\chapter{General Introduction}

\label{chap:Intro}

This thesis presents work investigating the role that recombination plays in the adaptive evolution of two eukaryotic microorganisms, \textit{Albugo candida} and \textit{Fragilariopsis cylindrus}.
Both of these organisms exist in environments that may be considered very dynamic.

In addition, methodological work was also conducted which implemented and tested software dedicated to making it easier to detect recombination in Next Generation Sequencing data. 
The software was also designed to help solve current methodological issues with distinguishing mosaic regions that are the result of hybridisation, and those that are the result of incomplete lineage sorting.

These works are presented in chapters \ref{chap:HC}, \ref{chap:Acandida}, and \ref{chap:diatom}. Each has a more detailed and focused introduction to the concepts specific to them.
It is the purpose of this chapter to provide an overview of the key concepts of population genetics that are relevant to this work and provide a wider context for the next three chapters.

In order to understand adaptive evolution, it is necessary to understand the five forces of population genetics and how they drive adaptive evolution.
What follows is an overview of the five fundamental forces of evolutionary change. Afterwards, an overview of hybrid zones, and an overview of current common Bioinformatics procedures and how they are used in population genetics analyses are presented.   

\section{The five forces of evolutionary change}

\subsection{Selection}

\label{sec:Selection}

Selection is the non-random, differential survival and reproduction of organisms as a result of their different phenotypes. 
A population contains many individuals, and these individuals vary in their genetic makeup; the population has genetic variance.
This genetic variation, in combination with some environmental effects, is the cause of the phenotypic variation in a population \parencite{Ridley2004}.
This phenotypic variation results in variation in survival, fecundity, and mating ability, and this ultimately determines whether an individual contributes any alleles to the next generation of that population:
Individuals may be better or worse at surviving, or may not be chosen by the opposite sex to mate \parencite{Hedrick2010}.
This can be expressed in terms of relative fitness. Relative fitness can be defined as the relative ability of different genotypes to pass on their alleles to future generations \cite{Charlesworth2010}.
Individuals with genotypes that have a higher relative fitness are expected to survive and pass their alleles on to the next generation, and so over several generations, those genotypes will increase in frequency in the population.

\subsubsection{The basic diploid model}

The basic diploid model of selection models how selection operates for a
single diploid locus, with two alleles.
The model assumes that there is random mating among individuals in a population, and that selection is operating identically for both sexes.
In this model, selection occurs through differences in viability and it is constant through space and time i.e. it acts on every individual in every generation, regardless of location.
Generations are discrete and non-overlapping and no mutation is occurring.
No gene flow or inbreeding occurs and the size of the population is infinite so there is no genetic drift \parencite{Charlesworth2010,Hedrick2010}.
Despite these assumptions it is still a very useful model to explore and describe how selection operates.

Assume there are two alleles of a single locus, denoted as $A_1$, and $A_2$.
With these two alleles, three possible diploid genotypes are possible.
Two of them are heterozygous: $A_1A_1$, and $A_2A_2$, and the third, $A_1A_2$ is heterozygous.
The relative fitnesses of $A_1A_1$, $A_1A_2$, and $A_2A_2$ are denoted as $w_{11}$, $w_{12}$, and $w_{22}$ respectively \parencite{Wright1937}.
The contribution of each genotype to the next generation can be calculated as the product of its relative fitness and its frequency prior to selection.
The contributions of $A_1A_1$, $A_1A_2$, and $A_2A_2$ are $p_0^2w_{11}$, $2p_0q_0w_{12}$, and $q_0^2w{22}$, where $p$ is defined as the frequency of $A_1$ and $q$ is defined as the frequency of $A_2$ \parencite{Charlesworth2010,Hedrick2010}.
Assuming Hardy-Weinberg allele proportions before selection, the mean fitness of the population is:

\begin{equation} \label{eq:meanfit}
\bar{w}=p_0^2w_{11}+2p_0q_0w_{12}+q_0^2w_{22}
\end{equation}

The frequency of a genotype after selection can be calculated by dividing its contribution by the mean fitness, for example, for $A_1A_1$ this is $p_0^2w_{11} / \bar{w}$.
The frequency of the alleles $A_1$ and $A_2$ after selection ($p_1$ and $q_1$) can be obtained by noting that the frequency of any of the two alleles is the sum of the frequency of the homozygous genotype and half the frequency of the heterozygous genotype \parencite{Charlesworth2010,Hedrick2010}.

\begin{subequations}
\begin{align}
p_1=\frac{p(pw_{11}+qw_{12})}{\bar{w}}\\
q_1=\frac{q(pw_{12}+qw_{22})}{\bar{w}}
\end{align}
\end{subequations}

The change in $q$ over one round of selection can be
defined as $\Delta q=q_1-q_0$. Substituting $q_1$ and simplifying the 
formula gives equation \ref{eq:freqchange} \parencite{Charlesworth2010,Hedrick2010}.

\begin{equation} \label{eq:freqchange}
\Delta q = \frac{pq(w_{2\cdot} - w_{1\cdot})}{\bar{w}}
\end{equation}

If $p$ or $q$ are $0$, then there can be no change in frequencies of that allele, 
as it is not present in the population \parencite{Charlesworth2010,Hedrick2010}.

\subsubsection{Different fitness relationships}

The formulas and quantities just described can be used to explore the effects of 
selection for different fitness relationships.
Different relative fitness values of $w_{11}$, $w_{12}$, and $w_{22}$ can be generated for different fitness relationships through the combination of two other coefficients: $s$ is the selection coefficient which measures the amount of selection against a homozygote, and $h$ is the level of dominance \parencite{Charlesworth2010,Hedrick2010}.
When $h$ is multiplied by $s$, this measures the amount of selection against a heterozygote \parencite{Charlesworth2010,Hedrick2010}.
These different fitness relationships are displayed in table \ref{table:fitarrays}.

\begin{table}
\centering
\caption{Fitness values for different fitness relationships, adapted from \cite{Hedrick2010}.}
\begin{tabular}{ P{4cm} P{2cm} P{2cm} P{2cm} }
\hline
Fitness Relationship & $A_1A_1$ & $A_1A_2$ & $A_2A_2$ \\
\hline
Recessive lethal & $1$ & $1$ & $0$ \\ 
Recessive detrimental & $1$ & $1$ & $1-s$ \\
Additive detrimental & $1$ & $1-(s/2)$ & $1-s$ \\
Purifying Selection & $1$ & $1-hs$ & $1-s$ \\
Positive Selection & $1+s$ & $1+hs$ & $1$ \\ 
Overdominance & $1-s_1$ & $1$ & $1-s_2$ \\
Underdominance & $1+s_1$ & $1$ & $1+s_2$ \\
\hline
\end{tabular}
\label{table:fitarrays}
\end{table}

A recessive lethal allele describes an allele which has a detrimental effect on the individual that is so severe it leads to death of the individual.
Examples of alleles with such effects include those that cause Tay-Sachs disease in humans \parencite{Myerowitz1997}.
Relative fitnesses for this situation are given in row one of table \ref{table:fitarrays}.
Using these values in the formulas \ref{eq:meanfit} and \ref{eq:freqchange} it can be demonstrated that the mean fitness of a population reaches $1$ when there is no $A_2$ allele in the population ($q=0$).
Furthermore, $\Delta q$ is largest when $q$ is large, and is smaller when $q$ approaches $0$ \parencite{Hedrick2010}.
Therefore, when the frequency of a recessive lethal is high it is purged by selection very quickly from the population.
The reason lethal recessive alleles are not purged as quickly when they are at low frequency is that they are present in heterozygotes, therefore the deleterious recessive alleles are not subject to differential selection \parencite{Hedrick2010}.

Some recessive alleles are not lethal, but they are detrimental to the fitness of an individual \parencite{Charlesworth2009a,Charlesworth2010}.
This type of fitness relationship is called a recessive deleterious relationship. 
Fitness values for this scenario are given in row 2 of table \ref{table:fitarrays}.
The selection coefficient ($s$) reflects how detrimental allele $A_2$ is. 
If $s = 1$, then $A_2$ would be a recessive lethal allele and selection would act as previously described.
Mean fitness is maximized when $q = 0$ and $\Delta q$ is greatest when $q_0=2/3$, and lower for smaller values of $q$ \parencite{Hedrick2010}.
Again this is because $A_2$ mostly occurs in individuals with a heterozygote genotype for low $q$.

Heterozygous individuals may have phenotypes that are intermediate to those of the two homozygotes.
If the phenotype of a heterozygote is exactly halfway between that of the homozygotes this is referred to as additivity.
Fitness values for additivity are shown on line 3 of table \ref{table:fitarrays}.
In this scenario $\Delta q$ is larger when both alleles are equally frequent in the population.
$\Delta q$ is greater at low value of $q$ than in the previous scenarios.
For low $q$, $A_2$ is mostly in heterozygotes, but the deleterious effects of $A_2$ are not masked in the heterozygotes when the fitness relationship is additive \parencite{Charlesworth2010,Hedrick2010}.

Alleles with additive and recessive effects have been discussed, but every possible level of dominance can be represented in the model with the $h$ coefficient.
Fitness relationships modeling different levels of dominance with $h$ are shown on lines 4 and 5 of table \ref{table:fitarrays}.
These fitness arrays describe purifying and positive selection.
Purifying selection acts to reduce the frequency of a detrimental allele in a population \parencite{Hedrick2010}.
In contrast, positive selection acts to increase the frequency of an alleles with effects that are beneficial in the current environment of a population.
In reality, selection acts in both positive and purifying roles simultaneously.
In both the models if $h = 0$ then the allele is recessive, if $h = 0.5$ it is additive, and if $h=1$ it is dominant \parencite{Charlesworth2010}. 
For positive selection, the fastest increase in $p$ occurs when the allele is dominant.
When the allele is additive, then $p$ still increases quickly.
However, it takes longer for $p$ to increase when $A_1$ is recessive.
At low frequencies, the beneficial $A_1$ allele typically occurs in heterozygotes, and as a recessive allele, selection does not act on it \parencite{Hedrick2010}.

In the scenarios previously described selection is a force acting to reduce genetic variation as an allele either increases or decreases in frequency in a population.
However, circumstances can cause selection to maintain allelic diversity in the population.
This is possible when the heterozygote individuals have a higher fitness than individuals of either of the two homozygote genotypes.
The phenomenon is called overdominance.
The fitness values for overdominance are listed on row 6 of table \ref{table:fitarrays}.
For selection to maintain both alleles in a population, $\Delta q$ must be equal to $0$ for some initial $q_0$ between $0$ and $1$ \parencite{Charlesworth2010}.
This is called the equilibrium frequency of $q$, and it is a function of both the selection coefficients for the two homozygotes.
When $q$ is below this equilibrium frequency, $\Delta q$ is positive.
When $q$ is above the equilibrium frequency, $\Delta q$ is negative.
Thus, as $q$ is perturbed away from this equilibrium $\Delta q$ shifts such that $q$ will return to this equilibrium \parencite{Hedrick2010}.
Therefore, both alleles are maintained in the population at a certain ratio.

Warfarin resistance in Rats is an example of heterozygote advantage.
Resistance was conferred to the rats by a dominant allele (R) at the VKORC1 locus. 
Individuals with one copy of R were resistant to Warfarin, but homozygous individuals had a much greater requirement for Vitamin K \parencite{Greaves1977}.
Heterozygote advantage has also been invoked to explain polymorphism at loci in the major histocompatibility complex (MHC) \parencite{Spurgin2010}.
Overdominance is also an explanation of hybrid vigour (heterosis) \parencite{Baranwal2012} and so this is of particular relevance to chapter \ref{chap:Acandida}, where the plausibility of of a generalist plant pathogen evolving through repeated hybridisation is discussed.

Underdominance describes the situation where heterozygous individuals have a lower fitness than homozygous individuals.
Fitness values for this relationship are shown on the last line of table \ref{table:fitarrays}. 
As with overdominance, there is an equilibrium frequency of $q$ for which $\Delta q = 0$. 
However, unlike overdominance, with underdominance, $\Delta q$ is positive above the equilibrium point and negative below it \parencite{Hedrick2010}.
Therefore the equilibrium is unstable, and allele frequencies move away from it, rather than towards it.

\subsubsection{Selection and dynamic environments}

The basic model of selection described effectively demonstrates the key concepts of when considering how selection acts.
However there are extensions to the model, for example, the model has been extended to account for more than two alleles.
Selection is the mechanism that causes adaptive evolution and directional selection and molecular evidence of past positive selection is abundant \parencite{Hoekstra2007}.
Most of the phenotypic characteristics we associate with species are thought to be the end result of selection, even if the adaptive function is not obvious. 

However, the efficiency of selection can be reduced:
Muller introduced the concept of Genetic Load. This is defined as the reduction in fitness from the maximum possible in a population \parencite{Davis2011}.
The principal factors causing genetic load are thought to be the presence of 
deleterious recessive mutations, maintained by a mutation-selection balance (see section \ref{sec:Mutation}), and the segregation of homozygotes when there is heterozygote advantage \parencite{Davis2011}.
Small isolated populations may suffer from genetic load because they can become fixed for detrimental alleles (see section \ref{sec:Gdrift}).

Evidence of balancing-selection; selection that maintains polymorphism like overdominance, is not as common \parencite{Bubb2006}, but there are scenarios in which selection does maintain polymorphism.
Selection varying in time and space, frequency dependent selection, and host-pathogen evolution, are three such models that are particularly pertinent to the research presented in this thesis as they model selection operating in a dynamic and changing environments.
A common aspect of these models is that they violate an assumption of the basic model: constant fitness \parencite{Charlesworth2010}. If constant fitness is not assumed, it can be shown that selection may maintain polymorphism even in absence of heterozygote advantage.

Relative fitnesses may depend on the frequency of of the different genotypes in the population. An allele may have a greater fitness when it is present in the population in low numbers and less fitness when it is present in larger numbers \parencite{Hedrick2010,Charlesworth2010}.
This is called negative frequency dependent selection.
Alternatively, an allele might increase in fitness as it increases in frequency \parencite{Hedrick2010,Charlesworth2010}.

Frequency dependent selection occurs where there are host-pathogen interactions.
Pathogens have genes known as virulence factors and effector genes, which enable them to infect a host.
New mutations in a host species that confer resistance to a pathogen will be at low frequencies but have a high selective advantage.
As a result, the allele will start to spread in the host population.
As the allele becomes more common, the pathogen will find fewer new hosts they can infect \parencite{Charlesworth2006,Frank1993,Seger1988}.
Therefore, pathogen numbers decrease and the advantage gained by being resistant diminishes.
Indeed, if there is a cost to maintaining the resistance it will even become detrimental.
This process also happens with the pathogens.
As hosts acquire resistance to a pathogen, pathogens with new mutations allowing them to infect previously resistant hosts will have a strong selective advantage.
The now susceptible host genotype will decrease in frequency, as the pathogen increases in frequency.
The selective advantage of the pathogen genotype is reduced and may even suffer a cost if it is less virulent than other pathogen genotypes at infecting other host genotypes \parencite{Charlesworth2006,Frank1993,Seger1988}.
Parasite genotype frequencies may therefore become balanced in a population, resulting in highly polymorphic genes in pathogens, such as antigenic genes in malaria, and effector genes in pathogens like \textit{Phytophthora infestans}\parencite{Morgan2007,Policy2001}.
This type of process, typically assuming gene-for-gene interactions between host and pathogen, leads to cycles of allele frequency changes in both the host and pathogen \parencite{May1983}.
This may be of particular importance to haploid pathogens which by definition, will not have their polymorphism maintained by heterozygote advantage, and may be subject to clonal interference which restricts levels polymorphism and the speed of adaptation \parencite{Gerrish1998}.

In addition to existing in balance, polymorphisms in a host or pathogen pathogen can become fixed due to their selective advantage, which can lead to a succession of fixation events in both host and pathogen as each is under selection pressure to counter adapt each others previous adaptations.
This is called an evolutionary arms race, and can lead to long term variability and rapid evolution of DNA sequences such as effector genes in plant pathogens, and R genes in plants, and accelerated molecular evolution (see chapter \ref{chap:Acandida}) \parencite{Brown2003,Charlesworth2006,Morgan2007,Paterson2010,Rose2004}.

Selection may maintain variation when there is enough temporal variation in relative fitnesses of different genotypes.
An allele with detrimental effects in one generation may confer an advantage in subsequent generations, should conditions change.
This scenario is pertinent to chapter \ref{chap:diatom} as the environment of \textit{Fragilariopsis cylindrus} is also temporally dynamic with seasonal changes such as freezing and thawing events.
Models of temporally changing fitnesses have shown that polymorphism is only maintained by selection under very strict conditions:
The geometric mean of fitness over $n$ generations for both homozygotes must be smaller than that of the heterozygote (equation \ref{eq:timefit}) \parencite{Haldane1963}. 

\begin{equation} \label{eq:timefit}
\Bigg( \prod_{i=1}^{n} w_{11\cdot i} \Bigg)^{1/n} < 1 > \Bigg( \prod_{i=1}^{n} w_{22\cdot i} \Bigg)^{1/n}
\end{equation}

This can be illustrated by considering two seasons, $A_1$ is advantageous in one season, and $A_2$ is advantageous in the other.
Fitness values in season one then are $1+s$, $1$, and $1-s$ for $A_1A_1$, $A_1A_2$, and $A_2A_2$ respectively.
In the second season, this is reversed and $A_1A_1$ has fitness $1-s$ and $A_2A_2$ has fitness $1+s$.
If the same number of generations is spent in each season, conditions for polymorphism are met, otherwise directional selection will result instead.
Such expectations from theory have been validated in experimental evolution studies with bacteria, where serial transfer regimes were used to emulate the effects of temporal variation \parencite{Rainey2000}.
Therefore, it seems that there is little evidence polymorphism is maintained by selection where fitnesses vary in time, without heterozygote advantage or frequency dependent selection.

\subsection{Genetic Drift and finite population sizes}
\label{sec:Gdrift}

Genetic drift is the chance changes in allele frequency that result from the random sampling of gametes from generation to generation in a finite population.

\subsubsection{The effect of drift}

Genetic drift has the same expected effect on all loci in a genome.
In a large population, on average only a small change in allele frequencies will occur as a result of genetic drift.
However, for smaller populations, genetic drift can cause larger fluctuations in allele frequencies and may even lead to the loss of fixation of alleles purely by chance alone \parencite{Hedrick2010,Charlesworth2010}.
Simulations of genetic drift reveal that small population sizes can cause replicate populations to drift apart in allele frequency.
The probability that an allele goes to fixation as a result of genetic drift in a finite population is proportional to its initial frequency, assuming differential selection is not occurring. $u(q) = q_0$
Over replicate simulated populations, the mean allele frequency does not change as a result of drift, but the distribution of allele frequencies over replicate populations does \parencite{Hedrick2010,Charlesworth2010}.
Therefore, drift is often examined by considering heterozygosity or the variance in allele frequencies of replicate populations. 

Consider a Wright-Fisher model population with $N$ (diploid) individuals and assume each contributes two haploid gametes to the next generation \parencite{Crow1970}.
For an offspring individual, the probability of drawing the same allele twice from the parents is $2N[1/(2N)]^2$.
The probability that they are different is $1-1/(2N)$.
Two alleles may also be identical by descent with probability:

\begin{equation} \label{eq:ibd}
f_{t+1} = \frac{1}{2N} + \bigg(1-\frac{1}{2N}\bigg)f_t
\end{equation}

This can be rewritten and the expected heterozygosity after $t$ generations derived:

\begin{subequations}
\begin{align}
H_{t+1} = \bigg(1-\frac{1}{2N}\bigg)H_t\\
\nonumber\\
H_{t} = \bigg(1-\frac{1}{2N}\bigg)^tH_0
\end{align}
\end{subequations}

This demonstrates that each generation, heterozygosity decreases at a rate that is an inverse function of the population size, and it is possible to calculate the expected heterozygosity after $t$ generations \parencite{Hedrick2010,Charlesworth2010}.
In addition, it is possible to relate observed, heterozygosity to the difference in expected heterozygosity and the variance in allele frequency. 
Taking account of this into the above equations and rearranging produces a formula for for the variation in allele frequencies at time $t$.
The formula shows that as the number of generations increases, the variance approaches a maximum value of $p0q0$.
This Wright-Fisher model assumes parents produce many gametes and zygotes, and of those $N$ are chosen to form the next generation. It is implicit that individuals are hermaphrodites and there is a small probability of self-fertilization.
The mean time until fixation of an allele due to drift depends on initial frequencies of the allele and the initial frequency of the allele \parencite{Hedrick2010,Charlesworth2010}.
As population size increases, the effect of drift becomes smaller as it takes more consecutive chance increases of an allele to fix it in the population.
For any given population size, the lower the initial allele frequency is, the longer it is for that allele to become fixed by drift.
With new neutral mutants, the expected time to fixation is four times the population size.

Explanations of drift often mention the population size $N$. However, in many situations the relevant value is the number of breeding individuals.
This may be very different from the census population size.
The concept of an effective population size makes it possible to consider an ideal population of size $N$ in which all parents have an equal expectation of being a parent of any individual progeny. i.e the Wright-Fisher model.
Effective population size can be measured by different methods: inbreeding, variance, and eigenvalue.
When a population remains the same size these measures are similar, however they may differ when populations are growing or shrinking \parencite{All2012,Waples2002}.
The effective population size can be influenced by the frequency of different sexes in a population, variance in reproduction, and varying numbers of individuals over several generations.

Bottlenecks and founder events are two specific cases where a population changes size significantly, influencing the effective population size.
A bottleneck describes a situation in which something occurs to drastically reduce the number of individuals which survive in a population, or otherwise get to contribute to the next generation of the population.
Typically, these are events such as natural disasters, overwintering, or epidemics.
A founder event describes a situation in which a population is started from a low number of individuals, for example individuals being carried to a new island or location.
In both cases, these events can cause large random changes in allele frequencies, resulting in lower heterozygosity and fewer alleles than the ancestral population.
The changes in allele frequencies resulting from bottlenecks and founder events generate genetic distance between two populations, equation \ref{eq:bndist} gives the standard genetic distance \parencite{Nei1987} after a bottleneck or founder event, where $t$ is the the number of generations the event lasted \parencite{Chakraborty1977}.

\begin{equation}
D_t = -\frac{1}{2}\ln\bigg(\frac{1-H_0}{1-H_t}\bigg)
\label{eq:bndist}
\end{equation}

\subsubsection{Drift and selection}

In a finite population, when there is no differential selection at a locus, an allele may become fixed or lost as a result of genetic drift.

In a population of infinite size, by definition there is no genetic drift, and selectively favored alleles increase in frequency and asymptotically approach fixation.
Detrimental alleles always reduce in frequency and approach loss.
In finite populations however, because of the effects of genetic drift, alleles may not always be fixed when they are favorable, and detrimental alleles may be fixed despite their detriment.
The probability of a favorable allele in a finite population is a function of the initial frequency of the allele, the extent to which selection favours that allele, and the size of the population.
\cite{Kimura1962} developed an equation that takes these factors to compute the probability of fixation of $A_1$ \parencite{kimura1971theoretical}.
The probability of fixation of an allele is a function of its initial frequency, the level of dominance, the effective population size, and its selective advantage.
The probability of fixation of an allele increases with increasing initial allele frequency and with increasing $Ns$ (the product of population size and selection coefficient).
When $Ns << 1$, this indicates that $s << 1/N$ and that the selective advantage of an allele is very low. In this case, changes in allele frequency are determined by drift.
When $Ns >> 1$, then $s$ is higher than $1/N$ and changes in allele frequency depend more on selection than on drift. 
The effect where alleles with low selection coefficients (and hence only slightly deleterious effects), may act as if they were neutral in small populations was first identified by \cite{Wright1931}, and described in terms of molecular evolution by \cite{Ohta1973}, who called it the nearly neutral model.

In a neutral situation in a finite population, the loss of heterozygosity is $1-1/(2N)$.
For any given balancing selection regime, the decay in heterozygosity can be defined as $H_{t+1} = (1-d)H_t$, where $d$ is the loss from unfixed allele frequency states and the gain for the absorbing states.
With no selection, $d$ is $1/(2N)$ i.e. the expression reduces to the neutral model of heterozygosity loss as a result of drift already described.
The ratio of decay for a neutral locus over one undergoing selection is called a retardation factor \parencite{Robertson1962}.
This factor is one when there is neutrality, but when $d$ is less than 1, then selection can slow the rate of fixation, or when $d > 1$, then selection is increasing the rate of fixation.
Even though selection may be balancing in an infinite population, in a finite population, less genetic variation may be retained than in a population with no selection.
Populations with heterozygote advantage, and unequal homozygote fitness values genetic variation is eliminated faster than in populations with neutrality.

\subsubsection{Impact of genetic drift}

Genetic drift needs to be considered when studying plant pathogens and organisms in very dynamic environments, as those populations may experience periodic population expansions or contractions.
Analysis of $Q_{ST}$ values of eight traits, and $F_{ST}$ values of eight neutral loci of the pathogenic fungus \textit{Rhynchosporium commune} revealed that the majority of the traits analysed were evolving according to stabilizing selection, although a trait for growth at 22 degrees centigrade was subject to diversifying selection and local adaptation \parencite{Stefansson2014}.
This was proposed to be due to the fact the pathogen exists in large rather homogeneous environments (i.e. homogeneous monoculture systems) where they mostly experience one host genotype, and therefore stabilizing selection plays a greater role than does drift or directional selection.
Furthermore, the cycles of frequency dependent selection and maintenance of diversity previously described would only be expected to occur if there were some allelic diversity - rare advantageous alleles - in the host.
Other plant pathogens have been significantly affected by changes in their population size.
For example, the global pandemic of \textit{Phytophthora infestans} was initiated by a single clone, which escaped to North America, and then to Europe, and then to the rest of the world \parencite{Goodwin1994}.
Analyses of RFLP loci of the pathogen \textit{Mycosphaerella graminicola} isolated from different locations, indicated that Mexican and Australian populations have low gene diversity \parencite{Zhan2003a}, consistent with founder events and genetic drift.
\cite{Steele2001} found that in Australia, \textit{Puccina striiformis} originates from a single founder event, the founding race identified corresponded to a race previously identified in Europe.

\subsection{Mutation}
\label{sec:Mutation}

Mutation is the alteration of the nucleotide sequence of the genome of an organism.
Mutations may be caused by errors in the DNA replication process, the insertions of a transposable element, chromosome breakage, and errors in meiosis.
Mutations may be be caused by chemicals or radiation, and these mutagens cause certain kinds of mutation, for example, ultraviolet light \parencite{Kozmin2005}.

Many spontaneous mutations may have detrimental effects as they affect the normal functioning of a gene.
However, many mutations have neutral or almost neutral effects, as they do not result in changes to proteins or otherwise change DNA only slightly \parencite{Grauer2000}.
A few mutations will confer beneficial effects and change proteins in a way that enhances the fitness of organism with the allele.
Of course whether or not a mutant is beneficial, deleterious, or neutral also depends on the environment \parencite{Grauer2000}.

Typically, the term mutation is often used to describe the smaller scale mutations which give rise to a new allele or sequence, larger alterations are often referred to as copy number variations, structural variations, or chromosomal abnormalities \parencite{Grauer2000,Hedrick2010}.
A mutation may involve a change in one nucleotide base, or it may involve changes in several nucleotides.
Short mutations where a few nucleotides are removed or inserted into the DNA sequence are called indels, which may cause a frame-shift mutation if the number of bases inserted or deleted is not a multiple of three.
The change affects the grouping of nucleotides into codons, affecting the reading frame or possibly introducing a stop codon.
Both base mutations and indels can cause a change in the protein produced transcription and translation of the gene \parencite{Grauer2000}.
Transposable elements are portions of DNA that can replicate themselves and move location within the genome of an organism \parencite{Grauer2000,Wicker2007}.
60\% of the maize genome and 15\% of the \textit{Drosophila melanogaster} genome consists of transposable elements \parencite{Biemont2006}.
Transposable elements have been characterized as junk, neutral, and agents of mutation and adaptation. Their behavior ranges from that of an extreme parasite, to that of a mutualist depending on the transposable element, the organism, and the area of the genome affected by one \parencite{Grauer2000}.

To understand genome evolution, mutation by gene duplication, deletion, and gene conversion are important.
Many genes such as globins, histones, enzymes, and MHC genes are members of multigene families. Such families are composed of several homologous genes, with similar function, and are often situated close together on a chromosome i.e. they are closely linked \parencite{Hedrick2010}.
Such multigene families are thought to have evolved through serial duplication of an ancestral gene.
Duplicate genes may cause dosage effects, or they may diverge, resulting in new functionality (neofunctionalisation), or they may retain only a subset of their original functionality  (subfunctionalisation). 
Further duplication and deletion of genes may occur through unequal crossing over or gene conversion \parencite{Grauer2000}.
Gene conversion is a process by which the nucleotide sequence of one allele or allele segment is replaced by a homologous sequence from another allele.
\cite{Voordeckers2012} demonstrated how the MALS family of genes, which code for proteins specialised to act on disaccharides, were likely to have evolved through duplication of an ancestral gene.
By reconstructing the ancestral genes, and testing their activity on different substrates, they found the ancestor was mostly active on maltose like substrates, but had some function on isomaltose like sugars.
Duplication and mutation resulted in a series of enzymes specialised for different substrates.
Many species of plant pathogens have genomes rich in both repeats and transposable elements \parencite{Raffaele2012,Kemen2012} and it is therefore suspected they play a role in the evolution of effector repertoires and can influence the expression of effectors \parencite{Whisson2012}.

Mutations may occur anywhere across the genome stochastically, according to a mutation rate, however there are hotspots in the genome which experience mutations more often than other regions. 
Research into \textit{E.coli} by \cite{Shee2012} has indicated such hotspots can be caused by double strand breaks in DNA which then lead to stress induced mutagenesis.
In the plant pathogen \textit{Neurospora crassa} duplicate sequences in DNA are detected and mutated during its sexual phase. 
The mechanism could cause linked duplicated genes to diverge further than unlinked ones \parencite{Cambareri1991}. 

It is often assumed that likelihood of mutation occurring is unaffected by selection, however there are exceptions.
In microorganisms it is known that mutator phenotypes can arise \parencite{Barrick2009}.
These increase the number of mutations occurring in the population, and facilitate the adaptation of large asexual populations to new conditions, even when the frequency of the mutators is low.
Such hyper-mutation can be genetically inherited, or can be transient.
Clinical isolates of many pathogens such as \textit{E. coli}, \textit{Streptococci spp.}, and \textit{Staphylococci spp.} have been found to contain high proportions of hypermutators \parencite{Jayaraman2011}.
Localization of the hyper-mutation to contingency genes or specific regions of the genome limit the risk of accumulating too many detrimental mutations through hyper-mutation \parencite{Jayaraman2011}.
In the case of an inheritable hyper-mutator allele, it may increase in frequency in a population through hitchhiking; it is physically linked to a selectively beneficial mutation it caused to occur \parencite{Giraud2001a}.
Several models demonstrating how hypermutators persist and succeed exist \parencite{Taddei1997,Tenaillon1999}, and Hyper-mutation is particularly beneficial strategy for microorganisms that are exposed to frequent and possibly unpredictable stresses (like pathogens) \parencite{DeVisser2002,Tanaka2003}.

Mutation is an important evolutionary force that generates the variation the other forces act on.
Several mechanisms in microbes and pathogens have been described through which such variation is generated, in addition to ways in which an organism might increase the rate at which this variation is generated during times of stress for for certain alleles.
Next the effects mutation has on populations and how it exists in balance with previously described forces is presented.

\subsubsection{Effect of mutations on populations}

The effect of mutation on population allele frequencies can be evaluated by assuming a forward-backward model of mutation \parencite{Hedrick2010}.
In this model, two types of allele are possible, a wild type allele ($A_1$) and a detrimental mutant ($A_2$).
In addition, mutation is reversible and may change wild type alleles to the mutant alleles (forward mutation), and the mutant alleles may mutate back to the wild type (backward mutation).
It is assumed forward mutations are more common than backward mutations.
This is because forward mutations are mutations that resulting in gene malfunction. It is assumed only a limited number of possible mutations could compensate for such forward mutations and result in a backward mutation.
Mutation from $A_1$ to $A_2$ occurs at a rate $u$, and mutation from $A_2$ to $A_1$ occurs at rate $v$.
The change in frequency of $A_2$ due to only mutation is $\Delta q = up - vq$.
This expression is linearly related to the allele frequency, but as $u$ and $v$ are small - mutation rates are typically low - mutation does not significantly affect the proportion of alleles in the population \parencite{Hedrick2010}.
An equilibrium is achieved if the forward and backward mutation rates are equal, and if $u$ is higher than $v$ then it is expected that the frequency of detrimental alleles would be higher than the wild type alleles \parencite{Hedrick2010}.
However this expectation is not realistic as it does not consider selection.

When mutations occur, they are the only copy in the entire population.
All the individuals in the population immediately after mutation are homozygous for the wild type allele ($A_1A_1$), and the mutant is heterozygous ($A_1A_2$).
This one heterozygous individual must mate with a homozygous individual.
The new mutant may be lost, only homozygous wild type offspring may be the outcome, or some offspring may be heterozygous with the new mutant allele.
If mating results in only one offspring, then there is a 50\% chance it is $A_1A_1$, and if $A_1A_2$ is the result, then there is still only one $A_1A_2$ individual in the population.
If mating results in two offspring, then the probability of loosing $A_2$ is halved.
So the frequency of $A_2$ in generations following the mutation event depends on how many progeny are the result of mating, and what type they are \parencite{Hedrick2010}.

The way in which purifying selection keeps detrimental alleles from increasing in frequency has previously been described.
The entire genome is subject to the opposite effects of mutation and selection, and the joint effects of mutation and selection is called the mutation-selection balance.
Assume that $A_2$ is deleterious and recessive, selection will act to reduce the frequency of $A_2$ as previously described.
Equation \ref{eq:mssdel} rewrites \ref{eq:freqchange} using the fitness values for a recessive deleterious allele from table \ref{table:fitarrays} \parencite{Hedrick2010}.

\begin{equation} \label{eq:mssdel}
\Delta q_s = \frac{sq^2p}{1 - sq^2}
\end{equation}

The increase in $q$ due to mutation then is $\Delta q_{mu} = up$, and assuming back mutation occurs at a low rate compared to $u$, as these forces have opposite effects, there is a point where they are at equilibrium (equation \ref{eq:mutseleq}) and the total change in allele frequency is $\Delta q = \Delta q_{mu} + \Delta q_s = 0$ \parencite{Hedrick2010}. 

\begin{equation} \label{eq:mutseleq}
up = \frac{sq^2p}{1-sq^2}
\end{equation}

If it is assumed that $q^2$ is small then equation \ref{eq:mutseleq} can be solved for the equilibrium genotype frequency ($q_e^2=u/s$), and the equilibrium allele frequency ($q_e=\sqrt{u/s}$).
This frequency is increased as a result of either higher mutation rate or lower selective disadvantage.
If the deleterious mutant were not completely recessive, the level of dominance $h$ can affect $q_e$. 
If $h$ is much larger than 0 and $q_e$ is small, then equilibrium allele frequency is approximately $u/hs$, and assuming $p$ is almost 1, the frequency of the mutant phenotype at equilibrium is $2u/s$.
As a general rule, as the level of dominance increases, the equilibrium allele frequency rapidly reduces \parencite{Hedrick2010}.

Mutations will contribute to the genetic load of a population, reducing its fitness from the maximum possible.
For a deleterious recessive mutation the load is $L = sq^2$ and at equilibrium $u = sq^2$, load is roughly equal to the mutation rate. 
If the deleterious mutant is dominant, then load becomes $L = 2u$ which shows that depending on the level of dominance, the mutation load can be between the mutation rate and twice the mutation rate.
If independence of fitness between loci is assumed, the fitness at locus $i$ may be defined as $\bar{w}_i$, and the overall fitness of the population is defined ad $\bar{w} = \bar{w}_i^n$.
The overall load is $L = 1-\bar{w}$. \cite{Crow1970} gave a formula for approximating the total load caused by mutation:

\begin{equation} \label{eq:totalload}
L \approx C \sum{u_i}
\end{equation}

Where $C$ is a constant between 1 and 2 and $u_i$ is the mutation rate of the locus $i$.

Joint consideration of mutation and drift forms the basis of the neutral theory.
The initial frequency of a new mutant $A_1$ in a population of $A_2$ alleles has an initial frequency of $p_0 = \frac{1}{2N}$.
The two alleles are neutral respective to each other, thus the probability of this mutant being fixed in the population is equal to its initial frequency as described in section\ref{sec:Gdrift}, and the probability of losing the mutant from the population is $u(q) = 1 - \frac{1}{2N}$.
Unless a population is very small, a new neutral mutation is likely to be lost from the population by drift alone (section \ref{sec:Gdrift}).
Loss of a mutant due to drift occurs more quickly than fixation.
This is because the change in frequency necessary to lose a new mutant is much smaller than that necessary to fix the new mutant.
\cite{kimura1971theoretical} formulated the average time to fixation and loss of a new mutant due to drift alone:

\begin{subequations}
\begin{equation}
T_1(p) = 4N_e
\end{equation}
\begin{equation}
T_0(p) = 2\bigg(\frac{N_e}{N}\bigg)\ln(2N)
\end{equation}
\end{subequations}

Assuming $N = N_e$ then the time to loss reduces to $2N/[\ln(2N)]$.
As a result, polymorphism is often transient.
Mutation acts to increase the number of alleles, whereas drift acts to reduce the number of alleles.
The properties of this equilibrium for the infinite alleles model were explored by \cite{Kimura1964} using the inbreeding coefficient. 
Recall that equation \ref{eq:ibd} gives the expected inbreeding coefficient.
This may be modified by the probability both alleles did not mutate:

\begin{equation} \label{eq:ftmut}
f_t=\bigg[\frac{1}{2N_e}+\bigg(1-\frac{1}{2N_e}\bigg)f_{t-1}\bigg](1-u)^2
\end{equation}

Setting $f_0 = 1$ (heterozygosity $H_0 = 0$) and $u = 10^{-5}$ and examining the change in heterozygosity over many generations for various values of $Ne$ it can be shown that it takes many generations, but eventually heterozygosity rises to approach an asymptotic value.
Furthermore, the asymptotic level of heterozygosity is greater when $N_e$ is greater.
As a consequence, when population size is small, the rise to the smaller asymptotic value occurs more quickly as genetic drift has a greater impact on the genetic variation change than does mutation \parencite{Kimura1964,Hedrick2010}.
If an equilibrium between mutation adding variation and drift eliminating variation from a population is assumed $f_t=f_{t-1}=f_e$, formula \ref{eq:ftmut} reduces to:

\begin{equation}
f_e \approx \frac{1}{4N_eu+1}
\end{equation}

Because $H = 1 - f$, equilibrium heterozygosity for the infinite allele neutral model can be obtained, where $\Theta = 4N_eu$:

\begin{equation}
H_e = \frac{\Theta}{\Theta + 1}
\end{equation}

This equilibrium is different to equilibrium previously described, as the allele frequencies are constantly changing, but the distribution of alleles remains mostly constant.
The above equation demonstrates that when $\Theta \approx 1$, then $H_e \approx 0.5$.
When $\Theta \gg 1$ then mutation primarily affects heterozygosity rather than drift and so $H_e$ is quite high.
The opposite is true, when $\Theta \ll 1$ then drift is the major determinant of heterozygosity and $H_e$ is low \parencite{Kimura1964,Hedrick2010}.

To examine the effect of a population bottleneck, assume a population starts at mutation-drift equilibrium. 
The population goes through a bottleneck and grows large once again \parencite{Nei2005}.
The expected genetic variation after the bottleneck depends on heterozygosity prior to the bottleneck, the size of the bottleneck, and the rate of increase after the bottleneck \parencite{Nei1975}.
The size of the bottleneck has a large effect on the number of alleles in a population, but average heterozygosity is mostly affected by the rate of growth after the bottleneck.
This is because whilst heterozygosity is reduced by the decrease in population size, when growth of the population after the bottleneck is slow, heterozygosity is lost each generation until it is large enough. 
Faster population growth rates allow populations to rebound as loss of heterozygosity only occurs during the first few generations following the bottleneck \parencite{Nei1975}.

Mutations can have selective effects.
When $s$ is less than $1/(2N)$ genetic drift is the stronger factor affecting allele frequency than selection and the mutant behaves neutrally, and deleterious mutants may become fixed as if they were neutral in small populations \parencite{Kimura1983,Lynch1990a,Lande1994}.
Over time, fitness declines which can lead to further reductions in population size, and hence mutations of increasingly detrimental effect behave as if they are neutral, and are more likely to be fixed.
Such a feedback is called mutation meltdown, and in theory could make small populations go extinct, \parencite{Lynch1995}.
\subsection{Population structure and gene flow}
\label{sec:Gflow}

Populations may be split into subpopulations due to geographical, ecological, or behavioral factors.
When a population is divided or there is more than one population, the amount of genetic exchange, or gene flow, between the subpopulations may differ between the different populations or subpopulation.
When gene flow is high between two populations or subpopulations, they are highly connected genetically and the amount of genetic variation between them is homogenized.
Conversely, when the amount of gene flow is low between populations or subpopulations, then genetic drift, selection, and mutation in the populations and subpopulations may lead to genetic differentiation \parencite{Charlesworth2010,Hedrick2010}.

Some types of movement of individuals like migrations will not actually result in gene flow, especially if the individual is only transiently passing through a population and does not breed with members of the population \parencite{Hedrick2010}.
Gene flow may be distinguished from simple migration as movement between groups that results in genetic exchange \parencite{Endler1977}.

When considering population subdivision it is often assumed that the subpopulations are always present.
Another view assumes they can die out, but they are repopulated from neighboring subpopulations, this is termed a metapopulation \parencite{Hanski1998}, and the dynamics of extinction and re-population make metapopulations differ from the basic concept of a subdivided population. What follows is a basic description of how gene flow effects populations using a simple genetic model, before the joint effects of gene flow and drift, and gene flow and selection are considered.

The continent-island model models a situation in which a large continent population is connected to a smaller island population \parencite{Charlesworth2010}.
The smaller island population receives migrants from a larger continent population.
The larger continent population is assumed to be large enough to render the effect of genetic drift negligible compared to the effect of gene flow.
Gene flow is assumed to have negligible effect on the source population. 
In this model, the proportion of migrants moving to the island is $m$, and the proportion of residents in the island population is $1 – m$.
The proportion of $A_2$ in the migrants coming from the continent is $q_m$ and the frequency of $A_2$ on the island before the gene flow is $q_0$ \parencite{Hedrick2010}.

Frequency of $A_2$ on the island after gene flow is calculated as:

\begin{equation} \label{eq:qislandflow}
q_1 = (1 - m)q_0+mq_m
\end{equation}

Formula \ref{eq:qislandflow} can be reduced to $q_0-m(q_0-q_m)$.

The change in frequency of q is then defined as:

\begin{equation} \label{eq:dqislandflow}
\Delta q = q_1 - q_0
\end{equation}

Formula \ref{eq:dqislandflow} reduces to $- m(q_0-q_m)$.

$q_m$ and $m$ are assumed to be constant \parencite{Hedrick2010}.
From these equations it is clear that $m=0$ then there is not migration from the continent to the island and so there is no change in allele frequency.
If $q_0 < q_m$ then the frequency of $q$ increases on the island.
If $q_0 > q_m$ the frequency decreases.
This indicates that there is a stable equilibrium freq of $A_2$ at $q_m = q_0$.

A general formula to calculate the frequency of $A_2$ for any generation $t$ has been derived as:

\begin{equation}
q_t=(1-m)^tq_0+[1-(1-m)^t]q_m
\end{equation}

In this formula, as $t$ increases the first term approaches $0$, and the second term approaches $q_m$ \parencite{Hedrick2010}.
Therefore eventually the frequency of $A_2$ in the island population converges to the frequency of $A_2$ in the continent population.
This is because gene flow is unidirectional, and therefore eventually all in the island population are descended from migrants.
Thus, the allele frequencies approach that of the continent i.e. the source of the migrants \parencite{Charlesworth2010}.
In this model, allele frequency changes at a maximum rate initially, and as the equilibrium is approached, it decreases.

A more general model assumes gene flow can occur among all parts of a structured population.
The model assumes there is $k$ different subpopulations, and that the proportion of individuals migrating from a subpopulation $i$ to another subpopulation $j$ is $m_{ij}$ \parencite{Hedrick2010}.
The values of $m_{ij}$ then can form a matrix called a backward migration matrix \parencite{Bodmer1968}.
In this matrix, the proportion of residents (i.e. not migrants) in each subpopulation $i$ are given by the diagonal values of the matrix (i.e. $m_{ii}$).
Each row of the matrix sums to $1$, because it describes the proportion of migrants coming into a population $i$ from the other $j$ populations.
For this model, the amount of allele $A_2$ in any subpopulation $i$ after gene flow is:

\begin{equation} \label{eq:generalflowqprime}
q\sp{\prime}_i=\sum_{j=1}^k m_{ij}q_{j}
\end{equation}

To process of allele frequency change over time can be described with matrix notation, where $M$ is the migration matrix, and $Q_t$ is the vector of allele frequencies in each population at generation $t$:

\begin{equation}
Q_{t+1} = MQ_t
\end{equation}

The above can be generalized for any $t$

\begin{equation}
Q_t=M^tQ_0
\end{equation}

\parencite{Hedrick2010}

In this model, as with the continent-island model previously described, after a period of time, allele frequencies in the subpopulations converge and approach an asymptotic value.
This value can be calculated with equation \ref{eq:generalflowqprime} using a migration matrix raised to a power of $t$ large enough that all elements have reached their asymptotic values.
This demonstrates the homogenizing effect gene flow has on populations when it is sustained for a period of time \parencite{Charlesworth2010,Hedrick2010}.


\subsubsection{Gene flow - drift balance}

Gene flow acts to homogenize populations as described above.
However populations are finite in size and so genetic drift will cause differences between the populations through the random fixation and loss of alleles.
The joint effects of gene flow and drift can be examined using a simple model of replicate island populations \parencite{Wright1940}.
Each island has $N$ individuals and receives a proportion of migrants each generation $m$, from a continent population.

When the gene flow between islands, and the population size of the islands are large the allele frequencies on the islands behave as previously described: they will converge to the frequencies of the continent.
However if population sizes are small, and the amount of gene flow is low, then the allele frequencies of the islands may differ from each other \parencite{Hedrick2010}.
So genetic drift causes allele frequencies in subpopulations to drift apart, whilst gene flow acts to homogenise the allele frequencies:
Take $N$ to be equal to $N_e$, the probability two alleles coalesce in generation $t-1$ is $1/(2N)$ and the probability that they do not is $1-1/(2N)$ \parencite{Hedrick2010}. 
The expected homozygosity in generation $t$ can be given as:

\begin{equation}
f_t=\frac{1}{2N}+\bigg(1 - \frac{1}{2N}\bigg)f_{t-1}
\end{equation}

This expression can be modified by the probability that both alleles are not migrants:

\begin{equation} \label{eq:ft}
f_t=\bigg[\frac{1}{2N}+\bigg(1 - \frac{1}{2N}\bigg)f_{t-1}\bigg](1-m)^2
\end{equation}

Assuming there is an equilibrium between gene flow homogenizing variation, and drift generating variation, then $f = f_t = f_{t-1}$ and $f = F_{ST}$, then

\begin{equation}
F_{ST} = \frac{(1-m)^2}{2N-(2N-1)(1-m)^2}
\end{equation}

\parencite{Hedrick2010}

$F_{ST}$ is the fixation index, a measure of genetic differentiation over subpopulations.
When $m = 0$ then $F_{ST} = 1$, and when $m=1$, then $F_{ST}=0$.
In other words when levels of gene flow are high, the genetic differentiation over subpopulations is low.
Ignoring the powers of two, and reducing the formula, $F_{ST}$ can be approximated:

\begin{equation} \label{eq:fstapprox}
F_{ST} \approx \frac{1}{4Nm + 1}
\end{equation}

Assuming $k$ subpopulations, the differentiation between populations can be given as

\begin{equation}
G_{ST} = \frac{1}{4Nm\bigg(\frac{k}{k-1}\bigg)^2 + 1}
\end{equation}

\parencite{Slatkin1995}.
In both equations, $Nm$ means the absolute number of migrants entering a population every generation.

$F_{ST}$ for any generation $t$ has been derived when $m = 0$

\begin{equation}
F_{ST(t)} = 1 - e^{t/2N}
\end{equation}

\parencite{Wright1943}.

The above expression is 0, when subpopulations are not very separated in early generations, and reaches a maximum of 1 as subpopulations are separated by drift.
The smaller the population size, the faster the subpopulations diverge due to drift.
The increase in $F_{ST}$ is fastest for the first $2N$ generations, after which time it approaches the maximum of $1$.

Iterating over formula \ref{eq:ft} allows examination of the rate of approach to equilibrium for different values of $N$ and $m$.
When population size is large and the amount of gene flow is large, then approach to equilibrium is fast, but when populations are large and gene flow is small, then the approach to equilibrium is slow \parencite{Hedrick2010}.

Population subdivision also affects the $N_e$ of populations.
For the island model:

\begin{equation}
N_e=\frac{kN}{1-F_{ST}}
\end{equation}

If $F_{ST}$ is low, then $N_e\approx kN$, but if gene flow is low then $N_e$ might be larger than $kN$ \parencite{Wright1943}.

\cite{Wright1940} gave an explicit method of estimating allele frequencies incorporating the effects of gene flow and drift for the island model.
Assuming the frequency of $A_2$ in migrants ($q_m$) is constant, when observing a large number of islands, their average allele frequency will be $q_m$, but depending on drift and gene flow, the distribution over the islands will vary.
The shape of the distribution depends on $4Nmq_m$ and $4Nm(1-q_m)$.
With large amounts of gene flow and large population sizes, the allele frequencies over the islands will not depart far from the mean \parencite{Hedrick2010}.
However, with lower $4Nmq_m$ and $4Nm(1-q_m)$, and if $q_m = 0.5$, then the distribution takes on a U shape: Drift plays a greater role in determining allele frequencies as alleles enter the islands by gene flow, and islands become temporarily fixed for either $A_2$, or instead for $A_1$.

Other models add an extra consideration by assuming different populations occupy positions in space, and that gene flow is restricted to certain routes or directions.
For example, the stepping stone model arranges populations in a one dimensional structure, and restricts gene flow to occurring only between populations that are adjacent in that one dimensional space \parencite{Hedrick2010}.
The effective population size of such a linearly divided population can be approximated as $N_e\approx kN$ \parencite{Maruyama1970}.
If populations are distributed across a landscape according to available habitat, then there may be distance-dependent gene flow between the populations.
In such case, expected patterns of genetic variation may be similar to the stepping stone models \parencite{Wright1943}.
It has been suggested that the amount of genetic divergence as estimated with $Nm$ or $F_{ST}/(1-F_{ST})$ should change as an inverse linear function of geographic distance ($Nm$), or as a linear function of geographic distance ($F_{ST}/(1-F_{ST})$) \parencite{Rousset1997}.

In metapopulations \parencite{Levins1969}, the dynamics of recolonization and extinction greatly influence $N_e$, the genetic variation present in the metapopulation, and the distribution of genetic variation over the subpopulations \parencite{Slatkin1977,Hedrick1997,Whitlock1997,Nunney1999}.
Many parameters can influence the rate at which genetic variation is lost, for example, the source of individuals recolonizing a previously extinct path might be from a single path, or a group of individuals from all other non-extinct patches.
A metapopulation with 20 patches, an infinite population size in each patch, and no gene flow except during recolonization, will have an effective size of 150 when recolonization of a patch is from a single female from another patch.
This low $N_e$ is due to the low number of founders in each recolonization \parencite{Hedrick1997, Hedrick2010}. 


\subsubsection{Gene flow - selection balance}

Gene flow and selection are often both important forces driving allele frequencies in a population.
Both forces are diverse in their effects on allele frequencies and so the interaction of the two forces can lead to complex results \parencite{Lenormand2002}.
Therefore, only a simple scenarios of selection and gene flow is introduced here.

Consider again the continent-island model, if the change in allele frequency due selection is $\Delta q_s$, and the change in allele frequency due to gene flow is $\Delta q_m$, then the change in allele frequency due to the joint effect of the two forces is $\Delta q=\Delta q_m + \Delta q_s$\parencite{Hedrick2010}.
Assuming the fitness values of $A_1A_1$, $A_1A_2$, and $A_2A_2$ are $1$, $1-s$, and $1-2s$ respectively, then $\Delta q$ can be expressed as $\Delta q = sq^2-(m+s)q+mq_m$ \parencite{Li1976}. 
When $\Delta q = 0$, there is equilibrium, and the equilibrium frequency is found by solving the quadratic equation.

\begin{equation}
q_e = \frac{1}{2s}\{(m+s)\pm[(m+s)^2-4msq_m]^{1/2}\}
\end{equation}

$A_1$ is favored if $s$ is positive, otherwise $A_2$ is favored \parencite{Hedrick2010}.
There are three main scenarios to consider, one where gene flow is much less, greater than, or equal to the absolute value of selection ($|s|$).
As $m$ increases with respect to $|s|$, genetic differentiation does not occur.
This is intuitive, as gene flow has a homogenizing effect as previously described, and with increasing $m$, its effects become more influential than the effects of selection, and the island's equilibrium frequency approaches that of the migrants coming from the continent \parencite{Li1976}.

Generally, the equilibrium frequency of an island depends on the selective advantage, the level of dominance on the island, and the amount of gene flow.
With high amounts of gene flow, even a favorable variant can be lost from an island, no matter its level of dominance.
This is called patch disappearance \parencite{Haldane1948}.
Thus gene flow is a force which limits selection and local adaptation \parencite{Lenormand2002}.


\subsubsection{Importance of gene flow}

Gene flow and genetic structuring significantly influence plant pathogen and marine plankton populations.
Gene flow is the force which introduces new virulence alleles into a new agricultural field, far from the source of original mutation.
Plant pathogen populations are often made up of one or a few clonal lineages which differentiate themselves from other populations (in chapter \ref{chap:Acandida} these are called 'races') \parencite{Koenig1997}.
In such populations, it may help instead to think of genotype flow rather than gene flow because of the high degree of linkage.
Genotype flow refers to the movement of entire genotypes between distinct populations.
Since many plant pathogens have an asexual stage and a sexual stage, both genotype flow and gene flow can occur.
An existing example of gene flow between plant pathogen populations is provided by \cite{Zhan2003a}, who demonstrated that \textit{Mycosphaerella graminicola} populations shared RFLP alleles, but no two populations had completely identical fingerprints, indicating that gene flow, but not genotype flow, was occurring.
An example of genotype flow is the global movement of a single clone of \textit{Phytophthora infestans}, out of Mexico in the 1840's as previously described.
Only one mating type escaped and spread globally, and as the organism has two mating types, sexuality was not possible until the other mating type escaped in the 1970's \parencite{Goodwin1992,Goodwin1994,Goodwin1995}.

There is substantial evidence of genetic structuring in marine plankton populations despite the high dispersal capacity of those organisms that might usually lead one to expect high levels of gene flow \parencite{Sildever2016}.
Oceanographic features like currents and eddies will create habitat heterogeneity which in turn leads to genetic population structuring \parencite{White2010,Sanford2011,Casabianca2012}, as do chemical and biotic properties of the oceans such as pH levels, temperature, salinity, and the presence or absence of predators and parasites \parencite{Cousyn2001,Decaestecker2007,Weisse2007,Yampolsky2014,Defaveri2014}.
All these factors may cause local adaptation resulting in population structuring.

\subsection{Recombination and linkage}

In the theory introduced so far, it has been assumed that alleles at a locus under consideration are transmitted independently of any alleles at any other loci.
This is called independent assortment \parencite{Hedrick2010}.
It was also assumed that the fitnesses of genotypes at any given locus were independent of the fitnesses of other genotypes at other loci.
However, this simplification is not valid in the majority of cases.
The transmission of genetic variants does not occur independently of other genetic variants.
This is because of linkage between genetic variants; variants are distributed across DNA molecules, and two variants situated on the same molecule are said to be physically linked.
The non-random association of alleles is called linkage disequilibrium (LD) \parencite{Lewontin1960}.
The amount of LD is generally an inverse function of the rate of recombination. Where recombination is the rearrangement of genetic material, especially by crossing over in chromosomes or by the artificial joining of segments of DNA from different organisms 

If one assumes a large randomly mating population has two alleles at one locus $A$ ($A_1$, $A_2$), and two alleles at a second locus $B$ ($B_1$, $B_2$), then four gametes or haplotypes are possible: $A_1B_1$, $A_1B_2$, $A_2B_1$, and $A_2B_2$.
The frequencies of these four haplotypes are denoted as $x_{11}$, $x_{12}$, $x_{21}$, and $x_{22}$.
The frequencies of each allele are $p_1=x_{11}+x_{12}$, $p_2=x_{21}+x_{22}$, $q_1=x_{11}+x_{21}$, and $q_2=x_{12}+x_{22}$ for $A_1$, $A_2$, $B_1$, and $b_2$, respectively \parencite{Lewontin1960}.

Assuming random association between alleles in gametes, then the frequency of each gamete is equal to the product of the frequencies of the alleles it is made of.
In other words $x_{11}=p_1q_1, x_{12}=p_1q_2, x_{21}=p_2q_1, x_{22}=p_2q_2$.
However, when this assumption does not hold and there is nonrandom association between alleles, the frequencies must be written as a function of these expected frequencies, with some deviation $D$ from the expectation.
Therefore, $x_{11}=p_1q_1+D, x_{12}=p_1q_2-D, x_{21}=p_2q_1-D, x_{22}=p_2q_2+D$.
$D$ is the LD parameter and it is a measure of the deviation from random association between alleles at different loci, $D = x_{11} - p_1q_1$ \parencite{Lewontin1960}.
In other words it is the observed frequency of a gamete, minus the expected frequency of the gamete.
By substituting values $p_1$ and $q_1$, $D$ may be written as:

\begin{equation}
D=x_{11}x_{22}-x_{12}x_{21}
\end{equation}

The gametes can be categorized as coupling or repulsion gametes.
Coupling gametes are those with alleles of the same subscript, and repulsion gametes are those with alleles with different subscripts.
$D$ then is the product of the frequencies of the two coupling gametes, minus the product of the frequencies of the repulsion gametes \parencite{Hedrick2010}.

From these four gametes, 10 genotypes are possible. The genotypes and their expected proportions are listed in Table \ref{table:linkagearray}.
These derivations make sense given that $A_1B_1/A_1B_1$ genotypes only produce $A_1B_1$ gametes, and that $A_1B_1/A_1B_2$ genotypes produce $1/2A_1B_1$ and $1/2A_1B_2$ gametes.
Double heterozygotes produce gametes different from the parental gametes due to recombination, e.g. $A_1B_2$ and $A_2B_1$ gametes can be produced by recombination of $A_1B_1/A_2B_2$ individuals.
The recombination rate is denoted as $c$ in Table \ref{table:linkagearray}.
$c$ ranges from 0 where there is no recombination between loci $A$ and $B$, to $0.5$ or independent assortment.
The frequency of each gamete in the next generation can be calculated the summing each of columns 3 to 6 in Table \ref{table:linkagearray}, the simplified way of working out such sums are given on the bottom line of the table, where $D_0$ is the initial amount of LD \parencite{Hedrick2010}.

\begin{table}
\caption{Expected frequencies for different gametes in a two-allele, two-locus system, adapted from \cite{Hedrick2010}.}
\resizebox{\textwidth}{!}{%
{\renewcommand{\arraystretch}{2}
\begin{tabular}{*{6}{c}}
\toprule
&& \multicolumn{4}{c}{Gametes of offspring} \\\cline{3-6}
Genotypes & Frequencies & $A_1B_1$ & $A_1B_2$ & $A_2B_1$ & $A_2B_2$ \\ 
\midrule
$A_1B_1/A_1B_1$ & $x_{11}^2$ & $x_{11}^2$ & $-$ & $-$ & $-$ \\ 
$A_1B_1/A_1B_2$ & $2x_{11}x_{12}$ & $x_{11}x_{12}$ & $x_{11}x_{12}$ & $-$ & $-$ \\
$A_1B_2/A_1B_2$ & $x_{12}^2$ & $-$ & $x_{12}^2$ & $-$ & $-$ \\
 
$A_1B_1/A_2B_1$ & $2x_{11}x_{21}$ & $x_{11}x_{21}$ & $-$ & $x_{11}x_{21}$ & $-$ \\
$A_1B_1/A_2B_2$ & $2x_{11}x_{22}$ & $(1-c)x_{11}x_{22}$ & $cx_{11}x_{22}$ & $cx_{11}x_{22}$ & $(1-c)x_{11}x_{22}$ \\ 
$A_1B_2/A_2B_1$ & $2x_{12}x_{21}$ & $cx_{12}x_{21}$ & $(1-c)x_{12}x_{21}$ & $(1-c)x_{12}x_{21}$ & $cx_{12}x_{21}$ \\
$A_1B_2/A_2B_2$ & $2x_{12}x_{22}$ & $-$ & $x_{12}x_{22}$ & $-$ & $x_{12}x_{22}$ \\

$A_2B_1/A_2B_1$ & $x_{21}^2$ & $-$ & $-$ & $x_{21}^2$ & $-$ \\
$A_2B_1/A_2B_2$ & $2x_{21}x_{22}$ & $-$ & $-$ & $x_{21}x_{22}$ & $x_{21}x_{22}$ \\
$A_2B_2/A_2B_2$ & $x_{22}^2$ & $-$ & $-$ & $-$ & $x_{22}^2$ \\\midrule

& $1$ & $x_{11}^\prime=x_{11}-cD_0$ & $x_{12}^\prime=x_{12}+cD_0$ & $x_{21}^\prime=x_{21}+cD_0$ & $x_{22}^\prime=x_{22}-cD_0$ \\\bottomrule

\end{tabular}}}
\label{table:linkagearray}
\end{table}

The amount of $D$ after one generation then is $D_1=x_{11}^\prime x_{22}^\prime-x_{12}^\prime x_{21}^\prime$.
After substitution and simplification this becomes $D_1=(1-c)D_0$, which is recursive and so can become
\begin{equation} \label{eq:ldatt}
D_t=(1-c)^tD_0
\end{equation}
with $D_t$ meaning the amount of LD after $t$ generations \parencite{Hedrick2010}.

With this formula we see that when there is no linkage ($c=0.5$) most disequilibrium is lost within a few generations, and with lower recombination rate, linkage is tighter as recombination does not break up associations between alleles as frequently, and so LD does not decay as fast.

To determine how long it will take for an initial amount of LD $D_0$ to decay to a given amount of LD $D_t$ the equation \ref{eq:ldatt} can be solved to give:
\begin{equation}
t=\frac{\ln(D_t/D_0)}{\ln(1-c)}
\end{equation}
\parencite{Hedrick2010}.

The measure of LD described is not the only one proposed \parencite{Hedrick1987,Lewontin1988,Devlin1995}.
To examine the extent of linkage equilibrium over chromosomes, the $r^2$ and $D^\prime$ are often used and the extent of LD measured varies with the estimated amount of recombination over chromosomes \parencite{Dawson2002}.

The rate of recombination $c$ is estimated as the proportion of recombinant gametes produced from a parent with a known gamete constitution \parencite{Hedrick2010}.
The amount of recombination can vary because of a few factors. 
Recombination can vary between the sexes, on different chromosomes, and between different regions on the chromosomes.
Regions of higher or lower levels of recombination than are expected are termed hot spots and cold spots \parencite{Arnheim2003,Kauppi2004}.
Patterns of LD can be used to try to putatively identify such hot and cold recombination regions and estimate rates of recombination \parencite{Stumpf2003,Ptak2004,Auton2007}, and many other methods of recombination detection in DNA sequences exist.
In chapter \ref{chap:HC} more methods for detecting recombination are discussed along with presentation of the HybridCheck software.

LD can be generated by multilocus selection.
For example, tightly linked members of a multigene family or supergene \parencite{bhl29158} may be under selection that generates linkage disequilibrium as each gene of the family is related in its adaptive function.
Multigene family members are created by serial gene duplication, followed by divergence through mutation, drift, and differential selection.
Therefore, they have historical association, but interacting effects between them may cause selection to maintain their association, keeping them in disequilibrium.
The MHC of vertebrates has properties of both supergenes and multigene families and is in linkage disequilibrium \parencite{Edwards1998,Beck2000}.

LD can be influenced by genetic drift \parencite{Hill1968,Ohta1969}. The effects of drift on LD can be considered by imagining the two-loci two-state model as four alleles at one locus.
Drift will alter the frequency of the gametes from generation to generation similar to that of a single loci model.
Thus, drift in small populations can lead to nonrandom associations between alleles at different loci \parencite{Hedrick2010}.
Recombination reduces the effect of drift, reconstituting some gametes.
The expected value of the LD measure $r^2$ for a given effective population size $N_e$ and a given rate of recombination between two loci $c$, can be expressed as:
\begin{equation} \label{eq:Ersq}
E(r^2)\approx\frac{1}{1+4N_ec}
\end{equation}
\parencite{Hill1968,Ohta1969}.

With large $N_ec$, $E(r^2)$ moves towards $0$, with smaller $N_ec$ $E(r^2)$ approaches $1$.
Just as with the single locus model, founder events and population bottlenecks can also influence LD.
If $N_e$ was small at some point in the past, the LD caused may still be present if the LD has not decayed \parencite{Hedrick2010}.
With large $N_ec$, equation \ref{eq:Ersq} is approximately
\begin{equation}
E(r^2)=\frac{1}{\rho}
\end{equation}
where $\rho$ is $4N_ec$ or the population recombination rate.
This is analogous to the population mutation rate $\theta = 4N_e\mu$ \parencite{Wall2000,Stumpf2003,Padhukasahasram2006}, and the expected amount of LD decreases as $\rho$ increases (assuming that drift is the only thing affecting LD) \parencite{Pritchard2001,Hedrick2010}.

Mutations may also generate low levels of LD, however recurrent mutation is unlikely to cause higher LD because as they are unlikely to occur associated with the same allele repeatedly, and any buildup of LD through mutation would occur more slowly than the process of recombination reducing LD \parencite{Hedrick2010}.
However, mutation coupled with recombination and gene flow are the source of new haplotypes in populations.
New genetic variants can increase in frequency by selection and drift, and hence all these factors in concert may create additional LD \parencite{Hedrick2010}.
Mutations may also break up LD if the mutation rate is high enough.
Assuming an allele $A_1$ which mutates to a disease allele $A_2$, creating a new gamete $A_2B_1$, if mutations from $B_1$ to any other $B$ allele occur at rate $\mu$, assuming no recombination, the association between a disease allele $A_2$ and $B_1$ is broken down.
This effect has been found to be especially significant for microsatellite loci, which are characterized by a high mutation rate relative to SNP and indel mutations \parencite{Payseur2008}.

Gene conversion can also affect LD, but typically only affects shorter DNA segments. Assume there is gene conversion around a gene $B$ in an $A_1B_1C_1/A_1B_2C_1$ individual, gene conversion could result in a $A_1B_2C_2$ gamete.
$B_1$ has been converted to $B_2$.
This would decrease LD between $A$ and $B$, and $B$ and $C$.
However, it would not affect LD between $A$ and $C$.
Many close sites do not have complete association, suggesting that reduction in LD is occurring through gene conversion \parencite{Ardlie2001}.
Note however that consecutive mutations can also explain the incomplete association between linked sites.
For example, consider three haplotypes $A_1B_1C_2$, $A_1B_2C_1$, and $A_1B_2C_2$ in a ~100bp fragment in a population sample.
This observation is consistent with recombination (between the \nth{1} and \nth{2} haplotype, with breakpoint between $B$ and $C$, creating the \nth{3} haplotype).
It is also consistent with gene conversion (e.g. a $C_1$ in an ancestral \nth{2} haplotype might have been converted by the $C_2$ of the \nth{1} haplotype, thereby creating a novel \nth{3} haplotype).
Finally, it is also consistent with mutation ($B_2 \rightarrow B_1$) in the ancestral \nth{3} haplotype ($A_1B_2C_2$), creating the \nth{1} haplotype, and a second mutation ($C_2 \rightarrow C_1$) in another copy of the ancestral $A_1B_2C_2$ haplotype (before or after the first mutation at any point in time) resulting in the \nth{2} haplotype.
In other words, and in contrast to \cite{Ardlie2001}, the observation that many close sites do not have complete association should not be taken as evidence for gene conversion because other evolutionary forces can explain this observation more plausibly.

Gene flow can also affect LD.
The amount of disequilibrium when two populations are mixed to produce a third can be expressed as
\begin{equation}
D=m_xm_y(p_{1\cdot x}-p_{1\cdot y})(q_{1\cdot x}-q_{1\cdot y})
\end{equation}
where $p_{1\cdot x}$ and $p_{1\cdot y}$ are the frequencies of of the $A_1$ allele in the two populations being mixed (population $x$ and population $y$), and $q_{1\cdot x}$ and $q_{1\cdot y}$ are the frequencies of the $B_1$ allele in the two populations \parencite{Hedrick2010}.
For LD to be generated, the frequencies of both loci must be different in the two populations.
The greater the difference, and the more equal the contributions are from each population, the more LD is generated \parencite{Hedrick2010}. 
    
Population subdivision reduces the rate of LD decay.
The reduction in heterozygotes in subdivided populations due to the Wahlund effect \parencite{Wahlund1928} reduces the opportunity to create recombinant gametes.
If the amount of gene flow is small, then it can determine the rate of LD decay \parencite{Nei1973}.
The amount of linkage disequilibrium has been expressed as $D\approx m/c$ \parencite{Barton2007} i.e. it is a balance between the rate of gene flow creating LD, and the rate of recombination reducing LD.
Since many factors including selection, drift, gene flow and mutation affect LD, it can be difficult therefore to attribute a cause of LD without historical knowledge or data.

Since alleles are linked and selection occurs at one or more loci we say that alleles have a genetic background \parencite{Hedrick2010}.
Multilocus phenomenon may explain some observations encountered in evolutionary genetics.
Apparent heterozygous advantage at a given marker locus may actually be caused by association of alleles at a linked locus to the alleles at the marker locus \parencite{Ohta1971}.
For example, \cite{VanOosterhout2009} proposed that the genetic variation at the MHC may be maintained by a linkage of the genetic load (or sheltered load) present at the peri-MHC region.
Recessive deleterious mutations associated with a given haplotype prevent the fixation of that haplotype in the population because these mutations would become expressed in homozygous state, reducing the fitness of that individual.
In other words, an MHC haplotype is “self incompatible” because it expresses its genetic load in homozygous state.
Assuming that each MHC haplotype has its “own” sheltered load of recessive deleterious mutations, this prevents their fixation in the population, and results in a balanced polymorphism \parencite{VanOosterhout2009}.
Recombination between MHC alleles is further reduced by negative epistasis, with selection operating against recombination because the recombinant haplotype are “incompatible” with both parental (non-recombinant) haplotypes.

Furthermore, changes in allele frequencies might be the result of selection acting on alleles at an associated locus to one being observed.
This can result in genetic hitchhiking, selective sweeps or background selection \parencite{Charlesworth2010}.
Genetic hitchhiking, previously described as the mechanism by which hypermutator alleles can be indirectly selected for in clonal populations (section \ref{sec:Mutation}), is possible because of linkage.
Neutral alleles can increase in frequency because of their association with a selected allele.
The magnitude of hitchhiking depends on the extent of linkage, inbreeding, and the initial amount of LD \parencite{Thomson1977,Hedrick1980,Kaplan1989}.
If there is no initial statistical association between the neutral and selected allele, there can be no hitchhiking, even if recombination rates are low.
To fully understand the effect of hitchhiking, the rate of change in frequency of the positively selected allele must be known \parencite{Hedrick2010}.
For example for a new advantageous recessive allele, initial increase in frequency due to selection will be low (see section \ref{sec:Selection}), providing time for recombination to reduce initial LD, and thus reducing the amount of expected hitchhiking of neutral alleles.
Hitchhiking can even create LD between two neutral loci if they are associated with a third selected locus \parencite{Thomson1977,Hedrick1980}.
One of the most important effects of hitchhiking is the reduction in heterozygosity of neutral or nearly neutral variation in areas of low recombination \parencite{Maynard-Smith1974}.
This is called a selective sweep and leaves a characteristic signature in genome sequences, which can be detected to provide evidence of recent selection \parencite{Hedrick2010}.

The projects presented in this thesis are concerned with how recombination has influenced the adaptive evolution of the two species studied. Specific aspects of recombination, sex and linkage relevant to each project are introduced in detail in subsequent chapters.
In the introduction to chapter \ref{chap:diatom} the advantages and disadvantages of recombination and sex are presented, to provide context to the question of why \textit{F. cylindrus} might have abandoned sex (as is hypothesized at the start of the study).
In chapter \ref{chap:Acandida} the evolutionary advantages and disadvantages of introgression and hybridisation is discussed in the context of results, and there multilocus concepts are important. 


\subsection{Hybrid zones, introgression, and hybrid speciation}
\label{sec:hzones}

Gene flow and recombination can result in so called hybrid zones, a physical location where hybrid offspring of two diverged taxa occur \parencite{Hewitt1985}.
A hybrid zone may form where divergence is occurring between adjacent populations of a species that was previously homogenous.
Parapatric and peripatric speciation is most likely to result in hybrid zones because the divergence and speciation is driven not by geographical isolation.
With parapatric speciation, changes in environmental conditions between the adjacent population can result in adaptations and reproductive isolation \parencite{Mayr1942}.
Founder events and random genetic drift play an important role during peripatric speciation.
Before reproductive isolation has evolved, ongoing gene flow and recombination between the two adjacent populations could result in a hybrid zone.
In this case, the hybrid zone is called a primary hybrid zone.
Hybrid zones may also form as a result of secondary contact between two populations of diverged taxa which were previously allopatric and had diverged as a result of geographic isolation.
In the latter case, partial pre-zygotic reproductive isolation has evolved, but this is broken down, for example due to changes in environmental conditions that could hinder conspecific mate choice.
It is often difficult to distinguish between primary and secondary hybrid zones \parencite{Endler1982}.

Such hybrid zones have a cline in the genetic composition across the zone from one of the parental forms to the other, as novel alleles from either side (that is either parental population) flow into the hybrid zone.
Such clines can either be gradual or stepped, and they can be observed by recording the frequency of diagnostic alleles for the parental populations, across the transect between the two parental populations \parencite{Hewitt1985}.
When quantifying the cline in this way, the frequency of diagnostic alleles is often characterized by a sigmoid curve, and the width of the cline is dependent on the ratio of hybrid survival to rate of recombination \parencite{Hewitt1985}.
In addition to a cline of genetic composition, hybrid zones often exhibit a higher variability in fitness within the zone.
In the middle of the cline hybrizymes may also be found.
Hybrizymes are rare alleles from both the parental taxa, which reach high frequencies where hybrids are formed, due to genetic hitchhiking of those alleles with alleles that contribute to hybrid fitness \parencite{Schilthuizen1999}. 

It is possible for alleles to flow back into the distinct parental populations through introgression (subsequent backcrossing of a hybrid individual breeding with a parental individual).
As a result, they appear to present a problem for the biological definition of a species if it is defined as “a population of (potentially) interbreeding individuals that produce fertile offspring”, however if the two parental populations remain identifiably distinct then there is no problem for the alternative concept of a species as “taxa that retain their identity, despite gene flow” \parencite{Mayr1942}.

When introgression occurs, each generation is less able to replace itself with genetically similar individuals as a result of the influx of alleles from across the hybrid zone, and this may lead to genetic assimilation and homogenization of the two parental populations \parencite{Robbins2014}.
However, hybridisation does not always lead to the merging and homogenizing of the two populations involved.
The different evolutionary outcomes of hybridization occur through different pathways in addition to introgression, like consequences of ecology such as hybrid vigour or hybrid inferiority \parencite{Edmands1999,Johansen-Morris2006,Rieseberg1998}.

Hybrid vigour can lead to a slowing of the growth rates of the two parental populations, because of the competition with the more fit hybrids \parencite{Slattery2008}.
But equally, if the increased hybrid fitness only applies in the hybrid zone, then a stable situation occurs in which the two parental populations are not threatened with assimilation, and instead hybrid speciation may occur, whereby hybridisation leads to hybrids which are reproductively isolated from either of the two parental populations.
Some hybrid zones can persist for thousands of years \parencite{White1966}.
This is possible as the hybrid zones are so called “tension-zones”.
In tension zones, there is a balance between ongoing hybridisation, dispersal of parental forms, and natural selection against hybrids (hybrid inferiority).
If those forces are in equilibrium, a stable tension zone persists \parencite{Bazykin1969}.
Recent studies identifying the signature of admixture across the genomes of native westslope cutthroat trout, and an invasive rainbow trout, revealed genome-wide selection against the invasive alleles, and that this was consistent across environments and populations \parencite{Kovach2016}.
It is important to note when considering the possible paths the evolution of a hybrid zone may take, that the different outcomes are not exclusive either/or scenarios:
For example, even though a hybrid zone may be maintained by negative selection acting on hybrids, and whilst some alleles from a parental population will be prevented from flowing into the other parental population as a result of negative selection, other alleles that are neutral or positively selected for may be able to flow across the hybrid zone and into the other population \parencite{Hewitt1985}.
Both of these processes are occurring at once, with the outcome varying across the genome, depending on the alleles.
In this way, a hybrid zone acts as a semi-permeable barrier to the flow of alleles.
Analysis of genetic and phenotypic variation across a hybrid zone of Antirrhinum, populations near the French-Spanish border is one such example demonstrating this \parencite{Whibley2006}:
The hybrid zone has a very steep cline in flower colour and morphology across the hybrid zone.
After crossing plant morphs to determine the contribution of the EL, ROS, and SULF alleles to magenta and yellow flower colouration, they used image analysis to score the levels of pigment in the plant and a principal component analysis on pixel scores together allowed the creation of a 3D genotypic space or “landscape” controlling flower colour \parencite{Whibley2006}.
Sequencing of natural samples across the hybrid zone allowed them to identify three main haplogroups.
One haplogroup was specific to the yellow morph, and the other two were found only in magenta morphs, the flower colour cline coincided with a cline in the frequency of these haplotypes.
The researchers then sequences loci not involved in flower colour determination, the PAL and DICH loci, which are linked to the ROS colour determination locus. 
They sequences PAL and DICH loci from 18 individuals either side of the hybrid zone.
They found PAL alleles fell into two distinct haplogroups, whilst DICH had no haplogroup structure \parencite{Whibley2006}.
Sequencing PAL and DICH alleles from individuals across the hybrid zone revealed no cline in the frequencies of these alleles, showing they are subject to different evolutionary forces.
These alleles also had no correlation with flower colour.
They concluded the distribution of the two alleles reflects historical gene flow, thus the hybrid zone is a barrier to alleles determining flower colour, as F2 hybrids are less fit according to their 3D fitness landscape, but other alleles are able to pass through \parencite{Whibley2006}.

Hybridization and introgression, is thought to occur in roughly 10\% of animal species and 25\% of plant species \parencite{Mallet2005a}.
Hybridization may lead to hybrid speciation, which is where new hybrid lineages become reproductively isolated from parental populations, and so are considered separate species.
Genomic studies have allowed  determination of the sizes of parental chromosomal blocks in introgressed populations and hybrid species \parencite{Buerkle2008,Morrell2005}, as they allow observation of associations among alleles of one species in the genetic background of another, indicating recent introgression.
Genome-wide studies of introgression and hybridisation have also supported the conclusions supported by the work of \cite{Whibley2006}, that there is variation in the amount of introgression across genomes, and so some regions of the genome are more permeable to foreign alleles than others \parencite{Martinsen2001,Macholan2007,Scotti-Saintagne2004,Turner2005,Yatabe2007}.

Substantial changes can occur to a genome immediately after hybridisation, such as gene loss or silencing, changes in expression of some genes \parencite{Adams2005}.
Analysis of three synthetic sunflower hybrids and three natural sunflower hybrid species has shown large karyotypic changes can occur over a handful of hybrid generations \parencite{Karrenberg2007,Lai2005}.
The natural hybrid species also exhibit increased genome sizes of up to nearly 50\% compared to the parental species \parencite{Baack2005}.
All species showed similar increases in genome size because of the proliferation of retrotransposons \parencite{Ungerer2006}.

The evolutionary consequences of hybridisation are complex.
F1 hybrids are often larger and more fit than their parents due to the effects of heterosis \parencite{Lippman2007}, due to either overdominance or the reciprocal complementation of deleterious alleles \parencite{ClarkCockerham1996}, this explains the establishment of hybrids but does not determine the longer term evolutionary success or failure of hybrids, which is more complex and is discusses in more detail in chapter \ref{chap:Acandida}.
In chapter \ref{chap:Acandida}, processes of hybridisation and introgression, and the evolutionary outcomes of such processes are discussed in more detail, and in the context of the work presented in that chapter, which focuses on the role of such processes in the adaptive evolution of a plant pathogen species as it adapted to many hosts.



\section{The role of Bioinformatics in population genetics}


Deoxyribonucleic acid was demonstrated as the genetic material by Oswald Theodore Avery in 1944 \parencite{Russell1988}.
Watson and Crick demonstrated its double helix structure composed of four nucleotide bases in 1953 \parencite{Watson1953}.
This led to the central dogma of molecular biology.
In most cases, genomic DNA defined the species and individuals, which makes the DNA sequence fundamental to the research on the structures and functions of cells.
Sequencing of genomes then is now an essential task  to complete, yielding essential data biologists need to understand biology and evolution of organisms.
The automated Sanger method was considered a first-generation sequencing technology \parencite{Sanger1975,Sanger1977}, and since then newer methods have been developed making sequencing cheaper and increasingly high throughput, these are referred to as next-generation sequencing (NGS) technologies \parencite{Goodwin2016}.

With the development of NGS technology, algorithms and tools for bioinformatics and evolutionary study have developed rapidly.
Here, I present a brief overview of the principles of several key bioinformatics tasks that population genetic studies with NGS data require. The processes below assume quality control of NGS reads is completed.


\subsection{Sequence Alignment}

An alignment of two sequences aims to discover or highlight how similar the two sequences are.
The concept of alignments is a natural one in settings where one sequence, changes over time into a second sequence, through a series of simple operations (called edit operations) like insertions of characters, deletions of characters, and a substitution of one character for another \parencite{makinen2015genome}.
It is unsurprising therefore, that alignments are a common first step in many evolutionary analyses. An alignment of the characters in two sequences, which have stayed the same over time, could be defined as the list of pairs of positions ($i$, $j$) such that the $i$th position in the first sequence is considered a match to the $j$th positions in the second sequence \parencite{makinen2015genome}.

In a practical setting, the two sequences (A \& B) are typically short homologous regions of the genomes of two different individuals, or species/taxa, and are considered to have evolved through a series of changes (edit operations), from some unobserved common ancestor \parencite{Lemey2009b}.
DNA sequence alignment algorithms typically require a scoring matrix which which they score potential alignments.
These matrices typically define scores for aligning any two characters in two sequences, and have some basis in biologically reality.
For example, the BLOSUM scoring matrix was derived from data of conserved regions of protein families \parencite{Lemey2009b}. 
The score of any given pairwise alignment is the sum of the scores that were assigned by the scoring matrix for each position of the alignment. 

A local alignment algorithm attempts to find the best alignments for sub-sequences of a query sequence with a reference sequence \parencite{Lemey2009b}.
Whereas global alignment algorithms attempt to find the best end to end alignment between a query sequence and a reference sequence.
Traditionally, pairwise sequence alignments were computed using dynamic programming algorithms such as the Needleman-Wunsch (global sequence alignment) \parencite{Needleman1970}, and the Smith-Waterman (local sequence alignment) algorithms \parencite{Smith1981}, but efficient and accurate techniques for sequence alignment is an active area of research, and so many advances, and different techniques and software packages have been developed.
Multiple alignment is the generalisation of pairwise sequence alignment to more than two sequences, this is a hard problem which becomes computationally unfeasible for many sequences without use of heuristics, such as the progressive alignment method, which first constructs a guide tree \parencite{Loytynoja2005}.

Sequence alignments can be used to align multiple gene or protein sequences together, align reads from high throughput sequencing platforms to a reference genome assembly \parencite{Li2009b}, or to align different genome assemblies together \parencite{Paten2011}.
In all cases these alignments may be used to run variant calling algorithms to infer the presence of mutations and structural alterations that are present in the genomes of different taxa, individuals, or populations, and can be used to genotype individuals, and compute population genetics and evolutionary analyses.


\subsection{Variant Calling}

Variant calling yields genotype data which may then be used in population genetics study.
Identification of SNPs (sometimes called Single Nucleotide Polymorphisms or simply mutations) can be done with a read pileup output after aligning reads to a reference genome \parencite{Li2009a,Li2011}.
If a position $j$ in the reference genome is covered by $n$ reads, and of those reads, $p$ per cent of them indicate that position $j$ is an ‘A’, and the rest indicate that position $j$ is a ‘G’, then it is possible to reason whether this is because the sample that was sequenced is polymorphic, or because of an alignment error or sequencing error \parencite{makinen2015genome}.
Such errors are easy to identify, as they are independent events, and as such exist in a very low frequency, because the probability of observing many errors in the same location decreases exponentially.
Therefore, so long as the sequencing is done to a sufficient depth of coverage, one can identify the polymorphic positions in a genome and rule out the errors with reasonable accuracy \parencite{makinen2015genome}.

Larger variants can also be detected from the read pileup.
If there is a deletion in the genome of the sample from which the reads were sequenced, then if it is larger than the error threshold in the alignment, then there should be regions of the read pileup where the reference is uncovered by reads \parencite{makinen2015genome}.
The region should have the same length as the deletion.
If there is an insertion in the genome of the sample from which the reads were sequenced, then if it is longer than the error threshold of the alignment, then in the pileup there would be a series of consecutive positions ($j$, $j+1$) for which no read covers both $j$ and $j+1$ \parencite{makinen2015genome}.
This is a simplistic approach to indel detection because in reality software implementations and algorithms also take into account errors, noise, base call qualities, and can have additional complexities such as utilizing data from many samples, and from linked sites. \parencite{Li2011,Nielsen2011,Mielczarek2016}.

Another approach to indel detection is to take advantage of sequencing technology platforms, which produce paired-end, or mate pair reads.
Sequencers can produce pairs of reads for each DNA molecule, one begins from one end of the molecule, and the other begins from the other end, and both extend towards the middle of the molecule.
When paired-end read pairs are aligned to a reference genome, they have an expected distance $k$ between them, this expected distance is known in advance according to the protocol used to prepare the DNA library for sequencing \parencite{makinen2015genome}.
It’s possible to compute the actual distance for each paired-end read pair, and then compute the mean and variance of those distances.
Once the mean distance $k'$ and variance is known, each paired-end read pair can be tested to see if its distance is significantly different to the average distance.
If the distance is significantly different an indel is inferred between those reads with length of $k-k'$ \parencite{makinen2015genome}. 


\subsection{Haplotype phasing}

Genotypes are the unordered combination of alleles at each site of an organism’s genome.
The haplotype are the sequences of alleles that have been inherited together from one parent.
For example, diploids possess two copies of each chromosome, therefore, in addition to being interested in which variants they possess (the genotype), one is also interested to know to which of a diploid’s two haplotypes each variant belongs – is the variant in the organism’s maternal copy of a DNA molecule, or is it in the paternal copy?
The process of identifying all the variants which are situated along the same haplotype of an organism is called haplotype phasing.
In an individual, variants which are clearly homozygous may be assigned to both haplotypes very simply as both haplotypes must possess them.

Given that when there are $N$ heterozygous sites in a sequenced DNA molecule, there are a total of $2N-1$ possible haplotypes, that could result in those haplotypes \parencite{makinen2015genome}.
Haplotype phasing was known to be a hard problem even before the development of high throughput sequencing technology.
However, advances have been made and several software packages now exists to perform this task.
The most accurate and widely used methods employ Hidden Markov Models to infer haplotypes \cite{makinen2015genome}.
For some time, a software implementation called PHASE was considered the superior method.
PHASE took ideas from coalescent theory about the joint distribution of haplotypes \parencite{Marchini2007,Marchini2010,Howie2011}.
PHASE was limited by its speed however and since the development of PHASE other methods implemented in packages like IMPUTE2 and SHAPEIT1 \& 2 have made improvements to the efficiency and accuracy of haplotype inference algorithms \parencite{Stephens2003,Delaneau2012,Delaneau2013,Delaneau2013a,OConnell2014}.

The flow of aligning high throughput sequencing reads to a reference, running variant calling and possibly haplotype inference, followed by downstream population genetic analysis on the genotype or haplotype data, is now a standard work-flow.
The choice of which software packages and algorithms should be used for each task can be a subjective decision which should aim to follow best-practice for each case in question.
For example, the best algorithm to use on human data, may not be the best one to use on an organism like wheat which has a radically different genome.






