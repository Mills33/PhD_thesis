\subsection{Mutation}
\label{sec:Mutation}

Mutation is the alteration of the nucleotide sequence of the genome of an organism.
Mutations may be caused by errors in the DNA replication process, the insertions of a transposable element, chromosome breakage, and errors in meiosis.
Mutations may be be caused by chemicals or radiation, and these mutagens cause certain kinds of mutation, for example, ultraviolet light \parencite{Kozmin2005}.

Many spontaneous mutations may have detrimental effects as they affect the normal functioning of a gene.
However, many mutations have neutral or almost neutral effects, as they do not result in changes to proteins or otherwise change DNA only slightly \parencite{Grauer2000}.
A few mutations will confer beneficial effects and change proteins in a way that enhances the fitness of organism with the allele.
Of course whether or not a mutant is beneficial, deleterious, or neutral also depends on the environment \parencite{Grauer2000}.

Typically, the term mutation is often used to describe the smaller scale mutations which give rise to a new allele or sequence, larger alterations are often referred to as copy number variations, structural variations, or chromosomal abnormalities \parencite{Grauer2000,Hedrick2010}.
A mutation may involve a change in one nucleotide base, or it may involve changes in several nucleotides.
Short mutations where a few nucleotides are removed or inserted into the DNA sequence are called indels, which may cause a frame-shift mutation if the number of bases inserted or deleted is not a multiple of three.
The change affects the grouping of nucleotides into codons, affecting the reading frame or possibly introducing a stop codon.
Both base mutations and indels can cause a change in the protein produced transcription and translation of the gene \parencite{Grauer2000}.
Transposable elements are portions of DNA that can replicate themselves and move location within the genome of an organism \parencite{Grauer2000,Wicker2007}.
60\% of the maize genome and 15\% of the \textit{Drosophila melanogaster} genome consists of transposable elements \parencite{Biemont2006}.
Transposable elements have been characterized as junk, neutral, and agents of mutation and adaptation. Their behavior ranges from that of an extreme parasite, to that of a mutualist depending on the transposable element, the organism, and the area of the genome affected by one \parencite{Grauer2000}.

To understand genome evolution, mutation by gene duplication, deletion, and gene conversion are important.
Many genes such as globins, histones, enzymes, and MHC genes are members of multigene families. Such families are composed of several homologous genes, with similar function, and are often situated close together on a chromosome i.e. they are closely linked \parencite{Hedrick2010}.
Such multigene families are thought to have evolved through serial duplication of an ancestral gene.
Duplicate genes may cause dosage effects, or they may diverge, resulting in new functionality (neofunctionalisation), or they may retain only a subset of their original functionality  (subfunctionalisation). 
Further duplication and deletion of genes may occur through unequal crossing over or gene conversion \parencite{Grauer2000}.
Gene conversion is a process by which the nucleotide sequence of one allele or allele segment is replaced by a homologous sequence from another allele.
\cite{Voordeckers2012} demonstrated how the MALS family of genes, which code for proteins specialised to act on disaccharides, were likely to have evolved through duplication of an ancestral gene.
By reconstructing the ancestral genes, and testing their activity on different substrates, they found the ancestor was mostly active on maltose like substrates, but had some function on isomaltose like sugars.
Duplication and mutation resulted in a series of enzymes specialised for different substrates.
Many species of plant pathogens have genomes rich in both repeats and transposable elements \parencite{Raffaele2012,Kemen2012} and it is therefore suspected they play a role in the evolution of effector repertoires and can influence the expression of effectors \parencite{Whisson2012}.

Mutations may occur anywhere across the genome stochastically, according to a mutation rate, however there are hotspots in the genome which experience mutations more often than other regions. 
Research into \textit{E.coli} by \cite{Shee2012} has indicated such hotspots can be caused by double strand breaks in DNA which then lead to stress induced mutagenesis.
In the plant pathogen \textit{Neurospora crassa} duplicate sequences in DNA are detected and mutated during its sexual phase. 
The mechanism could cause linked duplicated genes to diverge further than unlinked ones \parencite{Cambareri1991}. 

It is often assumed that likelihood of mutation occurring is unaffected by selection, however there are exceptions.
In microorganisms it is known that mutator phenotypes can arise \parencite{Barrick2009}.
These increase the number of mutations occurring in the population, and facilitate the adaptation of large asexual populations to new conditions, even when the frequency of the mutators is low.
Such hyper-mutation can be genetically inherited, or can be transient.
Clinical isolates of many pathogens such as \textit{E. coli}, \textit{Streptococci spp.}, and \textit{Staphylococci spp.} have been found to contain high proportions of hypermutators \parencite{Jayaraman2011}.
Localization of the hyper-mutation to contingency genes or specific regions of the genome limit the risk of accumulating too many detrimental mutations through hyper-mutation \parencite{Jayaraman2011}.
In the case of an inheritable hyper-mutator allele, it may increase in frequency in a population through hitchhiking; it is physically linked to a selectively beneficial mutation it caused to occur \parencite{Giraud2001a}.
Several models demonstrating how hypermutators persist and succeed exist \parencite{Taddei1997,Tenaillon1999}, and Hyper-mutation is particularly beneficial strategy for microorganisms that are exposed to frequent and possibly unpredictable stresses (like pathogens) \parencite{DeVisser2002,Tanaka2003}.

Mutation is an important evolutionary force that generates the variation the other forces act on.
Several mechanisms in microbes and pathogens have been described through which such variation is generated, in addition to ways in which an organism might increase the rate at which this variation is generated during times of stress for for certain alleles.
Next the effects mutation has on populations and how it exists in balance with previously described forces is presented.

\subsubsection{Effect of mutations on populations}

The effect of mutation on population allele frequencies can be evaluated by assuming a forward-backward model of mutation \parencite{Hedrick2010}.
In this model, two types of allele are possible, a wild type allele ($A_1$) and a detrimental mutant ($A_2$).
In addition, mutation is reversible and may change wild type alleles to the mutant alleles (forward mutation), and the mutant alleles may mutate back to the wild type (backward mutation).
It is assumed forward mutations are more common than backward mutations.
This is because forward mutations are mutations that resulting in gene malfunction. It is assumed only a limited number of possible mutations could compensate for such forward mutations and result in a backward mutation.
Mutation from $A_1$ to $A_2$ occurs at a rate $u$, and mutation from $A_2$ to $A_1$ occurs at rate $v$.
The change in frequency of $A_2$ due to only mutation is $\Delta q = up - vq$.
This expression is linearly related to the allele frequency, but as $u$ and $v$ are small - mutation rates are typically low - mutation does not significantly affect the proportion of alleles in the population \parencite{Hedrick2010}.
An equilibrium is achieved if the forward and backward mutation rates are equal, and if $u$ is higher than $v$ then it is expected that the frequency of detrimental alleles would be higher than the wild type alleles \parencite{Hedrick2010}.
However this expectation is not realistic as it does not consider selection.

When mutations occur, they are the only copy in the entire population.
All the individuals in the population immediately after mutation are homozygous for the wild type allele ($A_1A_1$), and the mutant is heterozygous ($A_1A_2$).
This one heterozygous individual must mate with a homozygous individual.
The new mutant may be lost, only homozygous wild type offspring may be the outcome, or some offspring may be heterozygous with the new mutant allele.
If mating results in only one offspring, then there is a 50\% chance it is $A_1A_1$, and if $A_1A_2$ is the result, then there is still only one $A_1A_2$ individual in the population.
If mating results in two offspring, then the probability of loosing $A_2$ is halved.
So the frequency of $A_2$ in generations following the mutation event depends on how many progeny are the result of mating, and what type they are \parencite{Hedrick2010}.

The way in which purifying selection keeps detrimental alleles from increasing in frequency has previously been described.
The entire genome is subject to the opposite effects of mutation and selection, and the joint effects of mutation and selection is called the mutation-selection balance.
Assume that $A_2$ is deleterious and recessive, selection will act to reduce the frequency of $A_2$ as previously described.
Equation \ref{eq:mssdel} rewrites \ref{eq:freqchange} using the fitness values for a recessive deleterious allele from table \ref{table:fitarrays} \parencite{Hedrick2010}.

\begin{equation} \label{eq:mssdel}
\Delta q_s = \frac{sq^2p}{1 - sq^2}
\end{equation}

The increase in $q$ due to mutation then is $\Delta q_{mu} = up$, and assuming back mutation occurs at a low rate compared to $u$, as these forces have opposite effects, there is a point where they are at equilibrium (equation \ref{eq:mutseleq}) and the total change in allele frequency is $\Delta q = \Delta q_{mu} + \Delta q_s = 0$ \parencite{Hedrick2010}. 

\begin{equation} \label{eq:mutseleq}
up = \frac{sq^2p}{1-sq^2}
\end{equation}

If it is assumed that $q^2$ is small then equation \ref{eq:mutseleq} can be solved for the equilibrium genotype frequency ($q_e^2=u/s$), and the equilibrium allele frequency ($q_e=\sqrt{u/s}$).
This frequency is increased as a result of either higher mutation rate or lower selective disadvantage.
If the deleterious mutant were not completely recessive, the level of dominance $h$ can affect $q_e$. 
If $h$ is much larger than 0 and $q_e$ is small, then equilibrium allele frequency is approximately $u/hs$, and assuming $p$ is almost 1, the frequency of the mutant phenotype at equilibrium is $2u/s$.
As a general rule, as the level of dominance increases, the equilibrium allele frequency rapidly reduces \parencite{Hedrick2010}.

Mutations will contribute to the genetic load of a population, reducing its fitness from the maximum possible.
For a deleterious recessive mutation the load is $L = sq^2$ and at equilibrium $u = sq^2$, load is roughly equal to the mutation rate. 
If the deleterious mutant is dominant, then load becomes $L = 2u$ which shows that depending on the level of dominance, the mutation load can be between the mutation rate and twice the mutation rate.
If independence of fitness between loci is assumed, the fitness at locus $i$ may be defined as $\bar{w}_i$, and the overall fitness of the population is defined ad $\bar{w} = \bar{w}_i^n$.
The overall load is $L = 1-\bar{w}$. \cite{Crow1970} gave a formula for approximating the total load caused by mutation:

\begin{equation} \label{eq:totalload}
L \approx C \sum{u_i}
\end{equation}

Where $C$ is a constant between 1 and 2 and $u_i$ is the mutation rate of the locus $i$.

Joint consideration of mutation and drift forms the basis of the neutral theory.
The initial frequency of a new mutant $A_1$ in a population of $A_2$ alleles has an initial frequency of $p_0 = \frac{1}{2N}$.
The two alleles are neutral respective to each other, thus the probability of this mutant being fixed in the population is equal to its initial frequency as described in section\ref{sec:Gdrift}, and the probability of losing the mutant from the population is $u(q) = 1 - \frac{1}{2N}$.
Unless a population is very small, a new neutral mutation is likely to be lost from the population by drift alone (section \ref{sec:Gdrift}).
Loss of a mutant due to drift occurs more quickly than fixation.
This is because the change in frequency necessary to lose a new mutant is much smaller than that necessary to fix the new mutant.
\cite{kimura1971theoretical} formulated the average time to fixation and loss of a new mutant due to drift alone:

\begin{subequations}
\begin{equation}
T_1(p) = 4N_e
\end{equation}
\begin{equation}
T_0(p) = 2\bigg(\frac{N_e}{N}\bigg)\ln(2N)
\end{equation}
\end{subequations}

Assuming $N = N_e$ then the time to loss reduces to $2N/[\ln(2N)]$.
As a result, polymorphism is often transient.
Mutation acts to increase the number of alleles, whereas drift acts to reduce the number of alleles.
The properties of this equilibrium for the infinite alleles model were explored by \cite{Kimura1964} using the inbreeding coefficient. 
Recall that equation \ref{eq:ibd} gives the expected inbreeding coefficient.
This may be modified by the probability both alleles did not mutate:

\begin{equation} \label{eq:ftmut}
f_t=\bigg[\frac{1}{2N_e}+\bigg(1-\frac{1}{2N_e}\bigg)f_{t-1}\bigg](1-u)^2
\end{equation}

Setting $f_0 = 1$ (heterozygosity $H_0 = 0$) and $u = 10^{-5}$ and examining the change in heterozygosity over many generations for various values of $Ne$ it can be shown that it takes many generations, but eventually heterozygosity rises to approach an asymptotic value.
Furthermore, the asymptotic level of heterozygosity is greater when $N_e$ is greater.
As a consequence, when population size is small, the rise to the smaller asymptotic value occurs more quickly as genetic drift has a greater impact on the genetic variation change than does mutation \parencite{Kimura1964,Hedrick2010}.
If an equilibrium between mutation adding variation and drift eliminating variation from a population is assumed $f_t=f_{t-1}=f_e$, formula \ref{eq:ftmut} reduces to:

\begin{equation}
f_e \approx \frac{1}{4N_eu+1}
\end{equation}

Because $H = 1 - f$, equilibrium heterozygosity for the infinite allele neutral model can be obtained, where $\Theta = 4N_eu$:

\begin{equation}
H_e = \frac{\Theta}{\Theta + 1}
\end{equation}

This equilibrium is different to equilibrium previously described, as the allele frequencies are constantly changing, but the distribution of alleles remains mostly constant.
The above equation demonstrates that when $\Theta \approx 1$, then $H_e \approx 0.5$.
When $\Theta \gg 1$ then mutation primarily affects heterozygosity rather than drift and so $H_e$ is quite high.
The opposite is true, when $\Theta \ll 1$ then drift is the major determinant of heterozygosity and $H_e$ is low \parencite{Kimura1964,Hedrick2010}.

To examine the effect of a population bottleneck, assume a population starts at mutation-drift equilibrium. 
The population goes through a bottleneck and grows large once again \parencite{Nei2005}.
The expected genetic variation after the bottleneck depends on heterozygosity prior to the bottleneck, the size of the bottleneck, and the rate of increase after the bottleneck \parencite{Nei1975}.
The size of the bottleneck has a large effect on the number of alleles in a population, but average heterozygosity is mostly affected by the rate of growth after the bottleneck.
This is because whilst heterozygosity is reduced by the decrease in population size, when growth of the population after the bottleneck is slow, heterozygosity is lost each generation until it is large enough. 
Faster population growth rates allow populations to rebound as loss of heterozygosity only occurs during the first few generations following the bottleneck \parencite{Nei1975}.

Mutations can have selective effects.
When $s$ is less than $1/(2N)$ genetic drift is the stronger factor affecting allele frequency than selection and the mutant behaves neutrally, and deleterious mutants may become fixed as if they were neutral in small populations \parencite{Kimura1983,Lynch1990a,Lande1994}.
Over time, fitness declines which can lead to further reductions in population size, and hence mutations of increasingly detrimental effect behave as if they are neutral, and are more likely to be fixed.
Such a feedback is called mutation meltdown, and in theory could make small populations go extinct, \parencite{Lynch1995}.