\subsection{Hybrid zones, introgression, and hybrid speciation}
\label{sec:hzones}

Gene flow and recombination can result in so called hybrid zones, a physical location where hybrid offspring of two diverged taxa occur \parencite{Hewitt1985}.
A hybrid zone may form where divergence is occurring between adjacent populations of a species that was previously homogenous.
Parapatric and peripatric speciation is most likely to result in hybrid zones because the divergence and speciation is driven not by geographical isolation.
With parapatric speciation, changes in environmental conditions between the adjacent population can result in adaptations and reproductive isolation \parencite{Mayr1942}.
Founder events and random genetic drift play an important role during peripatric speciation.
Before reproductive isolation has evolved, ongoing gene flow and recombination between the two adjacent populations could result in a hybrid zone.
In this case, the hybrid zone is called a primary hybrid zone.
Hybrid zones may also form as a result of secondary contact between two populations of diverged taxa which were previously allopatric and had diverged as a result of geographic isolation.
In the latter case, partial pre-zygotic reproductive isolation has evolved, but this is broken down, for example due to changes in environmental conditions that could hinder conspecific mate choice.
It is often difficult to distinguish between primary and secondary hybrid zones \parencite{Endler1982}.

Such hybrid zones have a cline in the genetic composition across the zone from one of the parental forms to the other, as novel alleles from either side (that is either parental population) flow into the hybrid zone.
Such clines can either be gradual or stepped, and they can be observed by recording the frequency of diagnostic alleles for the parental populations, across the transect between the two parental populations \parencite{Hewitt1985}.
When quantifying the cline in this way, the frequency of diagnostic alleles is often characterized by a sigmoid curve, and the width of the cline is dependent on the ratio of hybrid survival to rate of recombination \parencite{Hewitt1985}.
In addition to a cline of genetic composition, hybrid zones often exhibit a higher variability in fitness within the zone.
In the middle of the cline hybrizymes may also be found.
Hybrizymes are rare alleles from both the parental taxa, which reach high frequencies where hybrids are formed, due to genetic hitchhiking of those alleles with alleles that contribute to hybrid fitness \parencite{Schilthuizen1999}. 

It is possible for alleles to flow back into the distinct parental populations through introgression (subsequent backcrossing of a hybrid individual breeding with a parental individual).
As a result, they appear to present a problem for the biological definition of a species if it is defined as “a population of (potentially) interbreeding individuals that produce fertile offspring”, however if the two parental populations remain identifiably distinct then there is no problem for the alternative concept of a species as “taxa that retain their identity, despite gene flow” \parencite{Mayr1942}.

When introgression occurs, each generation is less able to replace itself with genetically similar individuals as a result of the influx of alleles from across the hybrid zone, and this may lead to genetic assimilation and homogenization of the two parental populations \parencite{Robbins2014}.
However, hybridisation does not always lead to the merging and homogenizing of the two populations involved.
The different evolutionary outcomes of hybridization occur through different pathways in addition to introgression, like consequences of ecology such as hybrid vigour or hybrid inferiority \parencite{Edmands1999,Johansen-Morris2006,Rieseberg1998}.

Hybrid vigour can lead to a slowing of the growth rates of the two parental populations, because of the competition with the more fit hybrids \parencite{Slattery2008}.
But equally, if the increased hybrid fitness only applies in the hybrid zone, then a stable situation occurs in which the two parental populations are not threatened with assimilation, and instead hybrid speciation may occur, whereby hybridisation leads to hybrids which are reproductively isolated from either of the two parental populations.
Some hybrid zones can persist for thousands of years \parencite{White1966}.
This is possible as the hybrid zones are so called “tension-zones”.
In tension zones, there is a balance between ongoing hybridisation, dispersal of parental forms, and natural selection against hybrids (hybrid inferiority).
If those forces are in equilibrium, a stable tension zone persists \parencite{Bazykin1969}.
Recent studies identifying the signature of admixture across the genomes of native westslope cutthroat trout, and an invasive rainbow trout, revealed genome-wide selection against the invasive alleles, and that this was consistent across environments and populations \parencite{Kovach2016}.
It is important to note when considering the possible paths the evolution of a hybrid zone may take, that the different outcomes are not exclusive either/or scenarios:
For example, even though a hybrid zone may be maintained by negative selection acting on hybrids, and whilst some alleles from a parental population will be prevented from flowing into the other parental population as a result of negative selection, other alleles that are neutral or positively selected for may be able to flow across the hybrid zone and into the other population \parencite{Hewitt1985}.
Both of these processes are occurring at once, with the outcome varying across the genome, depending on the alleles.
In this way, a hybrid zone acts as a semi-permeable barrier to the flow of alleles.
Analysis of genetic and phenotypic variation across a hybrid zone of Antirrhinum, populations near the French-Spanish border is one such example demonstrating this \parencite{Whibley2006}:
The hybrid zone has a very steep cline in flower colour and morphology across the hybrid zone.
After crossing plant morphs to determine the contribution of the EL, ROS, and SULF alleles to magenta and yellow flower colouration, they used image analysis to score the levels of pigment in the plant and a principal component analysis on pixel scores together allowed the creation of a 3D genotypic space or “landscape” controlling flower colour \parencite{Whibley2006}.
Sequencing of natural samples across the hybrid zone allowed them to identify three main haplogroups.
One haplogroup was specific to the yellow morph, and the other two were found only in magenta morphs, the flower colour cline coincided with a cline in the frequency of these haplotypes.
The researchers then sequences loci not involved in flower colour determination, the PAL and DICH loci, which are linked to the ROS colour determination locus. 
They sequences PAL and DICH loci from 18 individuals either side of the hybrid zone.
They found PAL alleles fell into two distinct haplogroups, whilst DICH had no haplogroup structure \parencite{Whibley2006}.
Sequencing PAL and DICH alleles from individuals across the hybrid zone revealed no cline in the frequencies of these alleles, showing they are subject to different evolutionary forces.
These alleles also had no correlation with flower colour.
They concluded the distribution of the two alleles reflects historical gene flow, thus the hybrid zone is a barrier to alleles determining flower colour, as F2 hybrids are less fit according to their 3D fitness landscape, but other alleles are able to pass through \parencite{Whibley2006}.

Hybridization and introgression, is thought to occur in roughly 10\% of animal species and 25\% of plant species \parencite{Mallet2005a}.
Hybridization may lead to hybrid speciation, which is where new hybrid lineages become reproductively isolated from parental populations, and so are considered separate species.
Genomic studies have allowed  determination of the sizes of parental chromosomal blocks in introgressed populations and hybrid species \parencite{Buerkle2008,Morrell2005}, as they allow observation of associations among alleles of one species in the genetic background of another, indicating recent introgression.
Genome-wide studies of introgression and hybridisation have also supported the conclusions supported by the work of \cite{Whibley2006}, that there is variation in the amount of introgression across genomes, and so some regions of the genome are more permeable to foreign alleles than others \parencite{Martinsen2001,Macholan2007,Scotti-Saintagne2004,Turner2005,Yatabe2007}.

Substantial changes can occur to a genome immediately after hybridisation, such as gene loss or silencing, changes in expression of some genes \parencite{Adams2005}.
Analysis of three synthetic sunflower hybrids and three natural sunflower hybrid species has shown large karyotypic changes can occur over a handful of hybrid generations \parencite{Karrenberg2007,Lai2005}.
The natural hybrid species also exhibit increased genome sizes of up to nearly 50\% compared to the parental species \parencite{Baack2005}.
All species showed similar increases in genome size because of the proliferation of retrotransposons \parencite{Ungerer2006}.

The evolutionary consequences of hybridisation are complex.
F1 hybrids are often larger and more fit than their parents due to the effects of heterosis \parencite{Lippman2007}, due to either overdominance or the reciprocal complementation of deleterious alleles \parencite{ClarkCockerham1996}, this explains the establishment of hybrids but does not determine the longer term evolutionary success or failure of hybrids, which is more complex and is discusses in more detail in chapter \ref{chap:Acandida}.
In chapter \ref{chap:Acandida}, processes of hybridisation and introgression, and the evolutionary outcomes of such processes are discussed in more detail, and in the context of the work presented in that chapter, which focuses on the role of such processes in the adaptive evolution of a plant pathogen species as it adapted to many hosts.
