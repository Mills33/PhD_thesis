As one of the five evolutionary forces, recombination fulfills both a “cleansing” role, as well as a role in generating genetic diversity.
Recombination ‘cleanses’ by separating deleterious mutations from their genomic background, increasing the efficacy of purifying selection and curtailing the continuous accumulation of deleterious mutations.
Recombination also plays a fundamental role in the repair of damaged DNA, and it can be a creative force, resulting in the formation of novel genotypes, haplotypes and alleles, thereby playing a key role in adaptive evolution.
By uniting beneficial mutations that exist at different loci in separate lineages, meiotic recombination during sex accelerates adaptive evolution.
Although recombination leaves a distinct signature or footprint in the genome of organisms, identifying this force can be difficult; subsequent recombination events tend to wipe out their past genomic footprints.
This thesis presents the development of a novel software package called HybridCheck, for the detection of genomic regions affected by recombination in Next Generation Sequence data, and the rapid molecular dating of recombination events.
Hybrid-Check was used to analyze recombination signal in different races of the plant pathogen \textit{Albugo candida}, a generalist obligate biotroph that infects Brassica plants.
I show that recombination facilitated occasional introgression and gene flow between host-specialized races.
This may have accelerated the rate of adaptive evolution, and possibly broadened the pathogen's host-range.
Finally, the genome of the polar diatom \textit{Fragilariopsis cylindrus} contains diverged alleles that are differentially expressed in different environmental conditions.
The hypothesis that ancient asexuality explains how the diverged alleles evolved is challenged, but not rejected, based on evidence of recombination presented in this thesis.
An alternative hypothesis is proposed: allelic divergence might have evolved despite the homogenizing effect of meiotic recombination as a result of very large effective population sizes and strong diversifying selection on \textit{F. cylindrus} in the polar environment.
